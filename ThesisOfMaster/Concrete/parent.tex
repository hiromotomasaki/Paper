% 論文の構成
% 表紙
% 概要
% 目次
% 第1章 緒論
% 第2章 運転支援システムの説明
% 第3章 乗降車データに基づいた最適配車問題の定式化
% 第4章 結論
% 謝辞
% 参考文献
% 付録

%卒業論文用雛形
\documentclass[a4j,12pt,oneside,openany]{jsbook}
% 英語なら以下を使う.
%\documentclass[a4j,12pt,oneside,openany,english]{jsbook}

\usepackage[dvipdfmx]{graphicx}
\usepackage[dvipdfmx]{hyperref}
\usepackage{pxjahyper}

\usepackage{amssymb}
\usepackage{amsmath}
\usepackage{latexsym}
\usepackage{overcite}


%jsbook を report っぽくするスタイルファイル
\usepackage{book2report}
%定理,補題,系,例題,証明などや英語用の定義がされています.
%自分なりにいじってください.
\usepackage{thesis}
% 具体的には以下のように定義されています.
% 英語の定理環境
%  \newtheorem{theorem}{Theorem}[chapter]
%  \newtheorem{lemma}{Lemma}[chapter]
%  \newtheorem{proposition}{Proposition}[chapter]
%  \newtheorem{corollary}{Corollary}[chapter]
%  \newtheorem{definition}{Definition}[chapter]
%  \newtheorem{example}{Example}[chapter]
%  \newtheorem{proof}{Proof}
% 日本語の定理環境
%  \newtheorem{theorem}{定理}[chapter]
%  \newtheorem{lemma}{補題}[chapter]
%  \newtheorem{proposition}{命題}[chapter]
%  \newtheorem{corollary}{系}[chapter]
%  \newtheorem{definition}{定義}[chapter]
%  \newtheorem{example}{例}[chapter]
%  \newtheorem{proof}{証明}
% 証明には番号をつけず,最後は Box で終わります.

% 英語で,見出しのフォントが気に入らなかったら
%\renewcommand{\headfont}{\bfseries}

% ページ数が少ないときはここを大きくしてごまかそう!!効果絶大!!
\renewcommand{\baselinestretch}{1.0}

\begin{document}

\renewcommand{\include}[1]{}
\renewcommand\documentclass[2][]{}

% 表紙
%卒業論文用雛形
\documentclass[a4j,12pt,oneside,openany]{jsbook}
% 英語なら以下を使う.
%\documentclass[a4j,12pt,oneside,openany,english]{jsbook}

\usepackage[dvipdfmx]{graphicx}
\usepackage[dvipdfmx]{hyperref}
\usepackage{pxjahyper}

\usepackage{amssymb}
\usepackage{amsmath}
\usepackage{latexsym}
\usepackage{overcite}


%jsbook を report っぽくするスタイルファイル
\usepackage{book2report}
%定理,補題,系,例題,証明などや英語用の定義がされています.
%自分なりにいじってください.
\usepackage{thesis}
% 具体的には以下のように定義されています.
% 英語の定理環境
%  \newtheorem{theorem}{Theorem}[chapter]
%  \newtheorem{lemma}{Lemma}[chapter]
%  \newtheorem{proposition}{Proposition}[chapter]
%  \newtheorem{corollary}{Corollary}[chapter]
%  \newtheorem{definition}{Definition}[chapter]
%  \newtheorem{example}{Example}[chapter]
%  \newtheorem{proof}{Proof}
% 日本語の定理環境
%  \newtheorem{theorem}{定理}[chapter]
%  \newtheorem{lemma}{補題}[chapter]
%  \newtheorem{proposition}{命題}[chapter]
%  \newtheorem{corollary}{系}[chapter]
%  \newtheorem{definition}{定義}[chapter]
%  \newtheorem{example}{例}[chapter]
%  \newtheorem{proof}{証明}
% 証明には番号をつけず,最後は Box で終わります.

% 英語で,見出しのフォントが気に入らなかったら
%\renewcommand{\headfont}{\bfseries}

% ページ数が少ないときはここを大きくしてごまかそう!!効果絶大!!
\renewcommand{\baselinestretch}{1.0}

\begin{document}

%%%%%%%%%%%% 題目 %%%%%%%%%%%%%%%%%%%%%%%%%%%%%%%%%%%%%%%%%%%%%%%%%%%%%%%
%%%%%%%%%%%% ここも適当に変えてもいいと思う %%%%%%%%%%%%%%%%%%%%%%%%%%%%%%%%%
\thispagestyle{empty}
\begin{center}
\vspace*{5mm}
{\Huge {\textbf{修} \hspace{12pt} \textbf{士} \hspace{12pt} \textbf{学} \hspace{12pt} \textbf{位} \hspace{12pt} \textbf{論} \hspace{12pt} \textbf{文}}}\\
\vspace{2cm}
{\Large 題\hspace{8mm}目}\\
\vspace{1cm}
\underline{\LARGE{タクシーの運転支援システム構築に関する研究}} \\
% \vspace{0.5cm}
% \underline{\LARGE{}} \\
\vspace{12mm}
{\large 指 導 教 員!}\\
\vspace{6mm}
\underline{\Large 潮 俊光 教 授}\\
\vspace{8mm}
{\large 報 告 者}\\
\vspace{6mm}
\underline{\Large 広本 将基}\\
\vspace{10mm}
{\Large 平成29年2月8日}\\
\vspace{14mm}
{\Large 大阪大学基礎工学研究科\\システム創成専攻社会システム数理領域\\博士前期課程}\\
\end{center}
\clearpage
\setcounter{page}{0}
タクシーの運転支援システムの構築と最適配車問題の定式化
\include{end}

\frontmatter

% 概要
%卒業論文用雛形
\documentclass[a4j,12pt,oneside,openany]{jsbook}
% 英語なら以下を使う.
%\documentclass[a4j,12pt,oneside,openany,english]{jsbook}

\usepackage[dvipdfmx]{graphicx}
\usepackage[dvipdfmx]{hyperref}
\usepackage{pxjahyper}

\usepackage{amssymb}
\usepackage{amsmath}
\usepackage{latexsym}
\usepackage{overcite}


%jsbook を report っぽくするスタイルファイル
\usepackage{book2report}
%定理,補題,系,例題,証明などや英語用の定義がされています.
%自分なりにいじってください.
\usepackage{thesis}
% 具体的には以下のように定義されています.
% 英語の定理環境
%  \newtheorem{theorem}{Theorem}[chapter]
%  \newtheorem{lemma}{Lemma}[chapter]
%  \newtheorem{proposition}{Proposition}[chapter]
%  \newtheorem{corollary}{Corollary}[chapter]
%  \newtheorem{definition}{Definition}[chapter]
%  \newtheorem{example}{Example}[chapter]
%  \newtheorem{proof}{Proof}
% 日本語の定理環境
%  \newtheorem{theorem}{定理}[chapter]
%  \newtheorem{lemma}{補題}[chapter]
%  \newtheorem{proposition}{命題}[chapter]
%  \newtheorem{corollary}{系}[chapter]
%  \newtheorem{definition}{定義}[chapter]
%  \newtheorem{example}{例}[chapter]
%  \newtheorem{proof}{証明}
% 証明には番号をつけず,最後は Box で終わります.

% 英語で,見出しのフォントが気に入らなかったら
%\renewcommand{\headfont}{\bfseries}

% ページ数が少ないときはここを大きくしてごまかそう!!効果絶大!!
\renewcommand{\baselinestretch}{1.0}

\begin{document}
\begin{abstract}
\par
電車電車電車電車電車電車電車電車電車電車電車電車電車電車電車電車電車電車電車電車電車電車電車電車
\par
新幹線新幹線新幹線新幹線新幹線新幹線新幹線新幹線新幹線新幹線新幹線新幹線新幹線新幹線新幹線新幹線
\end{abstract}
\include{end}

% 目次
\tableofcontents

\mainmatter
% 1章:緒論
%卒業論文用雛形
\documentclass[a4j,12pt,oneside,openany]{jsbook}
% 英語なら以下を使う.
%\documentclass[a4j,12pt,oneside,openany,english]{jsbook}

\usepackage[dvipdfmx]{graphicx}
\usepackage[dvipdfmx]{hyperref}
\usepackage{pxjahyper}

\usepackage{amssymb}
\usepackage{amsmath}
\usepackage{latexsym}
\usepackage{overcite}


%jsbook を report っぽくするスタイルファイル
\usepackage{book2report}
%定理,補題,系,例題,証明などや英語用の定義がされています.
%自分なりにいじってください.
\usepackage{thesis}
% 具体的には以下のように定義されています.
% 英語の定理環境
%  \newtheorem{theorem}{Theorem}[chapter]
%  \newtheorem{lemma}{Lemma}[chapter]
%  \newtheorem{proposition}{Proposition}[chapter]
%  \newtheorem{corollary}{Corollary}[chapter]
%  \newtheorem{definition}{Definition}[chapter]
%  \newtheorem{example}{Example}[chapter]
%  \newtheorem{proof}{Proof}
% 日本語の定理環境
%  \newtheorem{theorem}{定理}[chapter]
%  \newtheorem{lemma}{補題}[chapter]
%  \newtheorem{proposition}{命題}[chapter]
%  \newtheorem{corollary}{系}[chapter]
%  \newtheorem{definition}{定義}[chapter]
%  \newtheorem{example}{例}[chapter]
%  \newtheorem{proof}{証明}
% 証明には番号をつけず,最後は Box で終わります.

% 英語で,見出しのフォントが気に入らなかったら
%\renewcommand{\headfont}{\bfseries}

% ページ数が少ないときはここを大きくしてごまかそう!!効果絶大!!
\renewcommand{\baselinestretch}{1.0}

\begin{document}
\chapter{緒論}
\label{ch:1}
 \section{研究背景と目的}
 \label{sec:1_1}

\par
現在のタクシー業界は,高年齢化,低賃金,劣悪な労働環境という問題を抱えている.
流しのタクシーが空車で走行する距離を減らすことで,賃金の向上が達成出来るだけでなく,CO2排出量の削減にもつながる.
現状では,運転手の経験と勘から流し走行をしており,経験の浅い運転手への流し運転の支援は重要な課題である.
最近では,すべてのタクシーにGPSが装着されており,乗客を乗せた位置,一定走行時間・距離ごとの位置情報が無線でリアルタイムに会社へ送信されて,管理できるようになった.
また,名古屋ではタクシーの自動運転による実証実験が行われている.
こうした状況では,ビッグデータを活用して,顧客の発生予測をして,走行方向の支援を行うシステムを考えることは重要である.

\par
一方,インテリジェント交通システムでは,自動車の走行データをオンラインでセンシングできるプローブカーを用いた交通状況のモニタリング法が開発されてきた\cite{bib1}.
タクシーはプローブカートして重要な枠割を果たしているだけでなく,走行履歴から運転手の特性を推定することも可能となってきている\cite{bib2}.
さらに,携帯電話の位置情報や乗車履歴データから乗客の予測技術も急速に発達してきた\cite{bib3, bib4}.

\par
タクシーの運行状況のモニタリング,乗客の予測技術の発展に伴い,最近,タクシーの最適配車の研究が注目されている.
Seow等は,乗客からの呼びに対して乗客の待ち時間が最小となるようなタクシーの配車法を提案している\cite{bib5}.
Qu等は,空車で走行する距離の総和が最小になるような走行方法を提案している\cite{bib6}.
Miao等は,モデル予測制御を用いた最適配車法を提案し,サンフランシスコの市街地を対象に実証実験を行っている\cite{bib7, bib8, bib9}.

\par
本論文では,まず,タクシー乗務員の運行をサポートするシステムを提案する.
そして,タクシーの移動モデルを混合論理動的システムでモデル化し,合理的な運行を行うための制御器を提案する.
提案モデルでは,個々のタクシーに対して個別の制御入力を行うのではなく,部分領域に分けられた中にいるタクシーに対しては同一の制御入力を行う.
その点が文献\cite{bib7, bib8, bib9}とは異なる点である.

 % \par
 % タクシー業界は道路運送法の下で様々な規制がかけられていた.
 % しかし,2002年に道路運送法が改正され,規制緩和が行われた.
 % そのため,タクシー会社の新規参入が増え,都市部でのタクシーの供給が増えた.
 % また,名古屋ではタクシーの自動運転による実証実験が行われている.
 % こうした状況では,データに基づく配車や運行の方法を考えることは重要である.

 % \par
 % 一方,近年では通信環境が整備され,プロセッサーの性能が向上し,通信用チップが安価に入手できるようになった.
 % つまり,大量のデータを観測,収集し,解析することが容易になった.
 % そのため,サイバーフィジカルシステムの考え方に基づく制御が注目を浴びている.

 % \par
 % 本論文ではタクシー乗務員の運行をサポートするシステムと,合理的な運行をするための制御器を提案する.
 % また,その制御器の有効性を個々のドライバーが貪欲に運行した場合と比較を行うことによって示す.

 \section{論文の構成}
 \label{sec:1_2}

 \par
 本報告の構成について述べる.
 第2章では,私達が利用できるデータと提案するシステムについて述べる.
 第3章では, タクシーが営業を行う領域をいくつかの部分領域(セル)に分割し,セル間でのタクシーの移動モデルを混合論理動的システムでモデル化する.
 そして,そのモデルを用いたモデル予測制御法を提案しアプリケーションへの実装結果を示す.
 最後に,第4章では結論と今後の課題について述べる.

 \include{end}

% 2章:運転支援システムの説明
%卒業論文用雛形
\documentclass[a4j,12pt,oneside,openany]{jsbook}
% 英語なら以下を使う.
%\documentclass[a4j,12pt,oneside,openany,english]{jsbook}

\usepackage[dvipdfmx]{graphicx}
\usepackage[dvipdfmx]{hyperref}
\usepackage{pxjahyper}

\usepackage{amssymb}
\usepackage{amsmath}
\usepackage{latexsym}
\usepackage{overcite}


%jsbook を report っぽくするスタイルファイル
\usepackage{book2report}
%定理,補題,系,例題,証明などや英語用の定義がされています.
%自分なりにいじってください.
\usepackage{thesis}
% 具体的には以下のように定義されています.
% 英語の定理環境
%  \newtheorem{theorem}{Theorem}[chapter]
%  \newtheorem{lemma}{Lemma}[chapter]
%  \newtheorem{proposition}{Proposition}[chapter]
%  \newtheorem{corollary}{Corollary}[chapter]
%  \newtheorem{definition}{Definition}[chapter]
%  \newtheorem{example}{Example}[chapter]
%  \newtheorem{proof}{Proof}
% 日本語の定理環境
%  \newtheorem{theorem}{定理}[chapter]
%  \newtheorem{lemma}{補題}[chapter]
%  \newtheorem{proposition}{命題}[chapter]
%  \newtheorem{corollary}{系}[chapter]
%  \newtheorem{definition}{定義}[chapter]
%  \newtheorem{example}{例}[chapter]
%  \newtheorem{proof}{証明}
% 証明には番号をつけず,最後は Box で終わります.

% 英語で,見出しのフォントが気に入らなかったら
%\renewcommand{\headfont}{\bfseries}

% ページ数が少ないときはここを大きくしてごまかそう!!効果絶大!!
\renewcommand{\baselinestretch}{1.0}

\begin{document}
\chapter{運転支援システムの説明}
\label{ch:2}

 \section{緒言}
 \label{sec:2_1}
 \par
 本章では,タクシーの運転支援システムを実現するために企業との共同開発で作成したソフトウェアのシステム構成について説明を行う.
 ソフトウェアはスマートフォン上で実行され,乗務員に提示される.
 スマホを利用することでタクシー事業者はカーナビを購入する経費を抑えられるメリットがある.
 一方,スマホの画面は小さいので,タクシーの乗務員が使いやすいように搭載する機能を選定する必要がある.
 \section{システム構成}
 \label{sec:2_2}
\par
図\ref{fig:2_2_1}は実際に作成されたアプリ画面である.
\begin{figure}[hbtp]
 \centering
 \includegraphics[keepaspectratio, width=50mm]
 {Graphics/chapter2/201603311830.jpg}
 \caption{乗務員が見るアプリ画面}
 \label{fig:2_2_1}
\end{figure}
アプリの主な使い方は以下のとおりである.
まず,アプリを起動してアプリ上で提供される情報を参考にして流し営業を行い,乗客を乗せた場合は画面右上の緑のサークルをタップする.
すると,サークルが赤色に変化し,アプリ上でのタクシーのステータスが空車から実車に変化する.
乗客を下ろした場合は,もう一度サークルを押してタクシーのステータスを空車に戻す.
アプリはタクシーのステータスが変化した場合と一定時間・距離走行した場合に位置情報とステータス情報をサーバーに送信する.

\par
提供される情報は以下の5つである.
1つ目は営業領域内のイベント情報である.
「yahoo!路線情報」のサイトのHTML記述の中からデータ抽出を行い電車の遅延情報を取得し,画面上部に表示する.
電車の大幅な遅延が発生すると,タクシーやバスなどの他の公共交通機関を利用する人が確実に増えるため,この情報は重要である.
2つ目は過去の同曜日,同時刻付近に乗客を乗せた箇所を示すピン情報である.
ピンの色は3種類あり,青,黄,赤の順に乗車時間が10分未満,10分以上20分未満,20分以上であった乗降車記録を示している.
ピンが密集している箇所は乗客を獲得できる頻度が高いと言える.
3つ目は周囲で最もピンがある領域を示す方向を表す赤い三角形のオブジェクトである.
マップを拡大表示して走行していた場合に,周囲のピンを見ることができなくなるので,必要な情報である.
4つ目は推奨される走行方向を表す青い三角形のオブジェクトである.
すべてのタクシーが,ピンが集中している領域に利己的に集まると,それらの領域で供給過多が起きてしまい,全体として乗客獲得の機会を失ってしまう.
そこで,ピンの情報と空車分布の情報を利用して推奨する走行方向をサーバー側で計算する.
5つ目は古典的なニューラルネットワークにより予測した需要の中で需要が多い箇所を表すサークルの情報である.
雨が降るとタクシーを利用する人が増える.
しかし,ピンの表示は気象条件を考慮していない.
そこで,ニューラルネットワークの入力に天候や気温を含ませて,予測した需要の中で需要が多い箇所を表示することを考えた.

\par
図\ref{fig:fig2_2_2}は4つ目の推奨される走行方向の情報を導き,各乗務員に提示する流れを図化したものである.
\begin{figure}[hbtp]
 \centering
 \includegraphics[keepaspectratio, width=50mm]
 {Graphics/chapter2/architecture-crop.pdf}
 \caption{流しタクシーの運転支援システム}
 \label{fig:fig2_2_2}
\end{figure}
本システムでは,各タクシーから走行データと乗降車データをリアルタイムに受け取り,データベースに保存する.
表\ref{tab:tab2_2_3}は実際にデータベースに保存される乗降車データである.
\begin{table}[hbtp]
  \begin{center}
    \caption{データベースに保存される乗降車データ}
    \begin{tabular}{|l|l|l|l|l|l|} \hline
      乗車時刻 & 乗車緯度 & 乗車経度 & 降車時刻 & 降車緯度 & 降車経度 \\ \hline \hline
      2016-11-06 02:18:08 & 34.66128 & 135.50286 & 2016-11-06 02:30:21 & 34.64936 & 135.51944 \\
      2016-11-06 02:40:40 & 34.66781 & 135.50914 & 2016-11-06 02:47:48 & 34.67094 & 135.53897 \\
      2016-11-06 03:03:59 & 34.66925 & 135.50633 & 2016-11-06 03:11:49 & 34.67011 & 135.50619 \\ \hline
    \end{tabular}
    \label{tab:tab2_2_3}
  \end{center}
\end{table}
需要予測器では,表\ref{tab:tab2_2_3}の履歴データから乗客の発生分布を予測する.
最適化器では,モデル予測制御を用いて,この乗客予測と現在の空車のタクシーの分布から最適なタクシーの移動分布を求める.

\par
5つ目の情報を計算するためのニューラルネットワークは以下のような構成にした.
入力層は6つの情報が入力される.
1つ目は月情報のための4個のノードである.
1年を春(3月から5月),夏(6月から8月),秋(9月から11月),冬(12月から2月)の4つに分割して,春であれば春に割り当てられているノードのみに1が与えられ,それ以外のノードには0が与えられる.
2つ目は時刻情報のためのノードである.
24時間は1440分あるので1440個のノードを準備する.
各時刻で対応するノードのみに1が与えられる.
3つ目は祝日であるかどうかのノードである.
祝日であれば1が与えられ,そうでなければ0が与えられる.
4つ目は特別な日であるかどうかのノードである.
大阪のタクシー業界における特別な日とは5日,10日,20日,月末のことである.
タクシー事業者からの聞き取りで,これらの日は仕事でタクシーを利用する人が多く,それ以外の日と乗客数が異なることがわかったため,ノードに追加した.
5つ目は天候に関するノードである.
雨であれば0が与えられ,そうでなければ1が与えられる.
6つ目は気温に関する2つのノードである.
その日の最高気温と最低気温を入力ノードに加えた.
天候と気温の情報は気象庁のサーバーからXML形式またはJSON形式のデータで利用可能だが,イベント情報の時と同様に,サイトのHTML記述の中からデータ抽出を行った.
入力層のノードの総数は1449個である.
中間層は1層,ノード数は200個とした.
出力層は,タクシーが営業を行う対象領域をいくつかの分割領域に分けた時に,それぞれの分割領域の需要数を予測するために分割領域の数のノードを用意した.

 \section{結言}
 \label{sec:2_3}
 \par
 本章では,提案するアプリケーションのシステム構成について説明を行った.
 乗務員には周辺のイベント情報と,過去に乗客を乗せた箇所を示すピン情報と,周辺で最もピンがある領域を示す方向と,推奨される進行方向と,古典的なニューラルネットワークにより予測した需要の中で需要が多い箇所を表すサークルの情報を示す.
 イベント情報や需要予測に用いる気象情報などはインターネット上から取得する.
 インターネット上には気象情報のように時間によって変化するデータが過去のものも含めて公開されている.
 データの形式はXML形式やJSON形式のようにプログラムからの利用を考慮したものや,HTML記述のものもある.
 気象庁や国土交通省のような公的な機関が提供するデータだけではなく,SNSなどの手段を利用して得られたリアルタイムな情報をイベント情報として提供することも考えられる.
 また,需要予測器として古典的なニューラルネットワークを利用したが,学習データにスパーク性があるため過学習を起こす問題がある.
多変量時系列モデルのような他の予測方法もあるため,タクシー運転支援システムの実装に適した需要予測の方法を見つけることが,今後の研究課題である.

 \include{end}
% 3章:乗降車データに基づいた最適配車問題
%卒業論文用雛形
\documentclass[a4j,12pt,oneside,openany]{jsbook}
% 英語なら以下を使う.
%\documentclass[a4j,12pt,oneside,openany,english]{jsbook}

\usepackage[dvipdfmx]{graphicx}
\usepackage[dvipdfmx]{hyperref}
\usepackage{pxjahyper}

\usepackage{amssymb}
\usepackage{amsmath}
\usepackage{latexsym}
\usepackage{overcite}


%jsbook を report っぽくするスタイルファイル
\usepackage{book2report}
%定理,補題,系,例題,証明などや英語用の定義がされています.
%自分なりにいじってください.
\usepackage{thesis}
% 具体的には以下のように定義されています.
% 英語の定理環境
%  \newtheorem{theorem}{Theorem}[chapter]
%  \newtheorem{lemma}{Lemma}[chapter]
%  \newtheorem{proposition}{Proposition}[chapter]
%  \newtheorem{corollary}{Corollary}[chapter]
%  \newtheorem{definition}{Definition}[chapter]
%  \newtheorem{example}{Example}[chapter]
%  \newtheorem{proof}{Proof}
% 日本語の定理環境
%  \newtheorem{theorem}{定理}[chapter]
%  \newtheorem{lemma}{補題}[chapter]
%  \newtheorem{proposition}{命題}[chapter]
%  \newtheorem{corollary}{系}[chapter]
%  \newtheorem{definition}{定義}[chapter]
%  \newtheorem{example}{例}[chapter]
%  \newtheorem{proof}{証明}
% 証明には番号をつけず,最後は Box で終わります.

% 英語で,見出しのフォントが気に入らなかったら
%\renewcommand{\headfont}{\bfseries}

% ページ数が少ないときはここを大きくしてごまかそう!!効果絶大!!
\renewcommand{\baselinestretch}{1.0}

\begin{document}
\chapter{モデル予測制御 - 集中型最適化の場合 -}
\label{ch:3}
 \section{緒言}
 \label{sec:3_1}
 \par
 あ
 \section{あ}
 \label{sec:3_2}
 \par
時刻$k$のときの各領域(セル)$i(i=1, 2, \ldots , N)$での空車数を$x_i(k)$とおく.時間区間$[k,\ k+1)$の間に,領域$i$で実車になるタクシー数を$s_i(k)$,空車になるタクシー数を$e_i(k)$とおく.入力として,時間区間$[k,\ k+1)$に間に領域$i$から$j$へ移動する空車数$u_{i,j}(k)$とおく.ここで,$u_{i,i}(k)$は領域$i$にとどまる空車数である.すると各領域の空車数のダイナミクスは,
\begin{equation}  \label{eqn:system}
x_i(k+1)=x_i(k)-s_i(k)+e_i(k)+\sum_{j=1}^{N}\bigg (u_{j,i}(k)-u_{i,j}(k) \bigg)
\end{equation}
となる.

$d_i(k)$を時刻$k$のときの領域$i$で発生する需要とおく.
\begin{equation} \label{eqn:d}
d_i(k+1)=d_i(k)-s_i(k)+p_i(k)
\end{equation}
である.ただし,$p_i(k)$は時間区間$[k,\ k+1)$の間で新たに発生する需要であり,実データから予想される.制御理論的に言えば,外乱のようなもの.
時間区間$[k,k+1)$に間に空車になるタクシー数$e_i(k)$は
\begin{equation} \label{eqn:e_i}
e_i(k)=\beta_{i}r(k)
\end{equation}
ただし,$r(k)$は時刻$k$のときの実車の総数で,$\beta_{i}$は領域$i$で実車全体の中で領域$i$で空車になる割合である.つまり,時刻$k$のときの実車の中から$\beta_{i}r(k)$だけが時間区間$[k,\ k+1)$の間で領域$i$で空車になることを表し,$\beta_i$は実データから推定される.
実車の総数$r(k)$は,式(\ref{eqn:e_i})を用いると
\begin{eqnarray}
r(k+1) &=& r(k)+\sum_{i=1}^{N}\bigg( s_i(k)-e_i(k)\bigg) \nonumber \\
	&=& r(k)+\sum_{i=1}^{N}\bigg( s_i(k)-\beta_i r(k)\bigg) \nonumber  \\
	&=& \bigg( 1-\sum_{i=1}^{N}\beta_i\bigg) r(k)+\sum_{i=1}^{N} s_i(k) \label{eqn:r}
\end{eqnarray}
である.ここで,入力に関する制約として,各領域$i$について
\begin{equation} \label{eqn:con}
x_i(k)=\sum_{j=1}^{N}u_{i,j}(k)
\end{equation}
を与える.このことは,各領域において時刻$k$での空車をどこに移動させるかを決定し,それに沿って,空車が移動すると仮定していることになる.空車は制御入力に従って移動してから乗車できると仮定する.

このように入力(移動する空車数))を与えると,式(\ref{eqn:system}), (\ref{eqn:e_i}), (\ref{eqn:con})からシステムダイナミクスは
\begin{equation}  \label{eqn:simple_system}
x_i(k+1)=\beta_i r(k)-s_i(k)+\sum_{j=1}^{N} u_{j,i}(k)
\end{equation}
となる.ここで,式(\ref{eqn:r}), (\ref{eqn:con}), (\ref{eqn:simple_system})から
\begin{eqnarray*}
r(k+1)+\sum_{i=1}^{N}x_i(k+1) &=&  \bigg( 1-\sum_{i=1}^{N}\beta_i\bigg) r(k)+\sum_{i=1}^{N} s_i(k) + \sum_{i=1}^{N}\bigg( \beta_i r(k)-s_i(k)+\sum_{j=1}^{N} u_{j,i}(k) \bigg)\\
&=& r(k)+\sum_{i=1}^{N}x_i(k)
\end{eqnarray*}
この式は,タクシーの総数が時間で変動しないことを意味する.もともとタクシーの総数が変化するような制御をかけていないので,この結果は妥当といえる.
そこで,タクシーの総数を$L$とおくと
\begin{equation}
r(k)= L-\sum_{i=1}^{N}x_i(k)
\end{equation}
とおける.したがって,式(\ref{eqn:simple_system})は
\begin{equation}  \label{eqn:simple_system2}
x_i(k+1)=\beta_i \bigg(L-\sum_{j=1}^{N} x_j(k) \bigg) -s_i(k)+\sum_{j=1}^{N} u_{j,i}(k)
\end{equation}
さらに自動車の移動から必ず$0$にしなければならない$u_{i,j}(k)$
があるはずです.例えば,領域$i$に隣接しない領域$\ell$については
\begin{equation} \label{eqn:u_seiyaku}
u_{i, \ell}(k)=0 \qquad  \forall k
\end{equation}
とおいてもよい.このような入力は,式(\ref{eqn:simple_system})からはずしておいてよいでしょう.

以上より,タクシーの総数$L$を用いて,制約条件は以下のようになる.
\begin{eqnarray}
x_i(k+1) &=& \beta_i \bigg(L-\sum_{j=1}^{N} x_j(k) \bigg) -s_i(k)+\sum_{j=1}^{N} u_{j,i}(k) \\
d_i(k+1) &=& d_i(k)-s_i(k)+p_i(k)
\end{eqnarray}
ここで,時間区間$[k,\ k+1)$の間で実車になる台数$s_i(k)$については以下のように考える.まず,時刻$k$での空車が制御入力に従って,移動してから実車になりうるとする.このとき,時間区間$[k,\ k+1)$の間で実車から空車になるタクシーもすぐに実車になりうるので,時間区間$[k,\ k+1)$の間に領域$i$にいる実車になりうるタクシーの台数は$\beta_i r(k)+\sum_{j=1}^{N}u_{j,i}(k)$であり,実車になる台数はこの数を超えることはないので,
\begin{equation} \label{eqn:s2}
s_i(k)=\left\{
\begin{array}{ll}
 \alpha_i d_i(k) & \mbox{if }\beta_i \bigg( L-\sum_{j=1}^{N}x_j(k)\bigg) +\sum_{j=1}^{N}u_{j,i}(k) -\alpha_i d_i(k) \geq 0 \\
\beta_i \bigg( L-\sum_{j=1}^{N}x_j(k)\bigg) +\sum_{j=1}^{N}u_{j,i}(k) & \mbox{otherwise}
\end{array}\right.
\end{equation}
ただし,$\alpha_i$は領域$i$においてタクシーに乗車できる乗客の割合(定数・同定する必要あり.)である.
ここで,式(\ref{eqn:s2})が条件付きの式になっている.この式は以下のように変形すれは,システム全体はMLD(Mixed Logical Dynamical)システムになる.なお,MLDシステムの性質については松本さんの卒論を見てください.

まず,以下の論理変数$\delta_i(k) \in \{ 0,\ 1\}$を導入する.
\begin{equation} \label{eqn:delta}
\delta_i(k)=
\left\{ \begin{array}{ll}
1 & \mbox{if }h_i(x_1(k),\ \ldots, x_N(k),\ u_{1,i}(k),\ldots, u_{N,i}(k),d_i(k))\geq 0 \\
0 & \mbox{otherwise}
\end{array} \right.
\end{equation}
と定義する.ただし,
\begin{equation} \label{eqn:h}
h_i(x_1,\  \ldots, x_N,\ u_{1,i},\ldots, u_{N,i},\ d_i)=\beta_i \bigg( L-\sum_{j=1}^{N}x_j\bigg) +\sum_{j=1}^{N}u_{j,i}(k) -\alpha_i d_i
\end{equation}
である.このとき,松本さんの卒論の補題2.3(i)から,制約条件式(\ref{eqn:delta})は次の不等式制約条件になる.
\begin{eqnarray} \label{ineq:delta}
h^{inf}_{i}(1-\delta_i(k))\leq h_i(k) \leq h^{sup}_{i} \delta_i(k)+(\delta_i(k) -1) \epsilon
\end{eqnarray}
ただし$h_i(k)=h_i(x_1(k),\  \ldots, x_N(k),\ u_{1,i}(k),\ldots, u_{N,i}(k),\ d_i(k))$と略記し,$h^{inf}_{i}$, $h^{sup}_{i} \in R$は取りうる任意の$(x_1,\  \ldots, x_N,\ u_{1,i},\ldots, u_{N,i},\ d_i)$に対して$h^{inf}_{i}\leq h_i(x_1,\  \ldots, x_N,\ u_{1,i},\ldots, u_{N,i},\ d_i)\leq h^{sup}_{i}$であり,$\epsilon \in R_{+}$は十分に小さな正の実数である(この数は$i$に独立でよいでしょう.,$h^{inf}_{i}$, $h^{sup}_{i}$も$i$に独立に指定してもよいかもしれません.).

式(\ref{eqn:s2})は以下のように変形できる.
\begin{eqnarray}
s_i(k) &=& \delta_i(k) \alpha_i d_i(k)+(1-\delta_i(k))S_i(k) \\
	&=& -\delta_i(k)h_i(k)+S_i(k)
\end{eqnarray}
ただし,
\[
S_i(k)=\beta_i \bigg( L-\sum_{j=1}^{N}x_j(k)\bigg) +\sum_{j=1}^{N}u_{j,i}(k)
\]
である.
ここで,
\begin{equation} \label{eqn:z}
z_i(k)=-\delta_i (k) h_i(k) \qquad \bigg( =\delta_i(k) ( -h_i(k)) \bigg)
\end{equation}
とおくと,
\begin{equation} \label{eqn:s_z_x}
s_i(k)=z_i(k)+S_i(k)
\end{equation}
となる,ここで,松本さんの卒論の補題2.3(ii)から,式(\ref{eqn:z})は,
\begin{eqnarray}
& -h^{sup}_{i} \delta_i(k) \leq z_i(k) \leq -h^{inf}_{i} \delta_i(k) & \label{ineq:z1}\\
& -h_i(k)+h^{inf}_i (1-\delta_i(k)) \leq z_i(k) 
   \leq -h_i(k)+h^{sup}_i (1-\delta_i(k)) & \label{ineq:z2}
\end{eqnarray}
と変換できる.

これらの式は線形なので,求める最適化問題は混合整数計画問題(制約条件式が線形の等式または不等式で書かれている.)として定式化できる.
 \section{結言}
 \label{sec:3_3}
 \par
 あ
 \include{end}
% 4章: 結論
%卒業論文用雛形
\documentclass[a4j,12pt,oneside,openany]{jsbook}
% 英語なら以下を使う.
%\documentclass[a4j,12pt,oneside,openany,english]{jsbook}

\usepackage[dvipdfmx]{graphicx}
\usepackage[dvipdfmx]{hyperref}
\usepackage{pxjahyper}

\usepackage{amssymb}
\usepackage{amsmath}
\usepackage{latexsym}
\usepackage{overcite}


%jsbook を report っぽくするスタイルファイル
\usepackage{book2report}
%定理,補題,系,例題,証明などや英語用の定義がされています.
%自分なりにいじってください.
\usepackage{thesis}
% 具体的には以下のように定義されています.
% 英語の定理環境
%  \newtheorem{theorem}{Theorem}[chapter]
%  \newtheorem{lemma}{Lemma}[chapter]
%  \newtheorem{proposition}{Proposition}[chapter]
%  \newtheorem{corollary}{Corollary}[chapter]
%  \newtheorem{definition}{Definition}[chapter]
%  \newtheorem{example}{Example}[chapter]
%  \newtheorem{proof}{Proof}
% 日本語の定理環境
%  \newtheorem{theorem}{定理}[chapter]
%  \newtheorem{lemma}{補題}[chapter]
%  \newtheorem{proposition}{命題}[chapter]
%  \newtheorem{corollary}{系}[chapter]
%  \newtheorem{definition}{定義}[chapter]
%  \newtheorem{example}{例}[chapter]
%  \newtheorem{proof}{証明}
% 証明には番号をつけず,最後は Box で終わります.

% 英語で,見出しのフォントが気に入らなかったら
%\renewcommand{\headfont}{\bfseries}

% ページ数が少ないときはここを大きくしてごまかそう!!効果絶大!!
\renewcommand{\baselinestretch}{1.0}

\begin{document}
\chapter{結論}
\label{ch:5}
 \par
 本論文では,タクシーの走行を支援するシステムを提案し,実装を行った結果を示した.
 タクシーの移動モデルを混合論理動的システムを用いて表し,モデル予測制御を応用して混合整数計画問題を解くことで空車タクシーの最適移動分布を求める方法である.
 実装にあたっては,大阪のタクシー事業者から実データをいただき,実際の空車分布から実証実験を行った.

\par
今後の課題としては,混合整数計画問題の計算の並列化,最適解の近似などによる計算時間の短縮が挙げられる.

\par
現在,車の自動運転に関して多くの研究が行われている.
いずれ,タクシーの自動運転が実用化される可能性があるため,タクシーの最適配車問題は今後も様々な研究がなされていくと考える.
本研究が,これらの研究の発展に寄与することを期待する.

 \include{end}

% 謝辞
%卒業論文用雛形
\documentclass[a4j,12pt,oneside,openany]{jsbook}
% 英語なら以下を使う.
%\documentclass[a4j,12pt,oneside,openany,english]{jsbook}

\usepackage[dvipdfmx]{graphicx}
\usepackage[dvipdfmx]{hyperref}
\usepackage{pxjahyper}

\usepackage{amssymb}
\usepackage{amsmath}
\usepackage{latexsym}
\usepackage{overcite}


%jsbook を report っぽくするスタイルファイル
\usepackage{book2report}
%定理,補題,系,例題,証明などや英語用の定義がされています.
%自分なりにいじってください.
\usepackage{thesis}
% 具体的には以下のように定義されています.
% 英語の定理環境
%  \newtheorem{theorem}{Theorem}[chapter]
%  \newtheorem{lemma}{Lemma}[chapter]
%  \newtheorem{proposition}{Proposition}[chapter]
%  \newtheorem{corollary}{Corollary}[chapter]
%  \newtheorem{definition}{Definition}[chapter]
%  \newtheorem{example}{Example}[chapter]
%  \newtheorem{proof}{Proof}
% 日本語の定理環境
%  \newtheorem{theorem}{定理}[chapter]
%  \newtheorem{lemma}{補題}[chapter]
%  \newtheorem{proposition}{命題}[chapter]
%  \newtheorem{corollary}{系}[chapter]
%  \newtheorem{definition}{定義}[chapter]
%  \newtheorem{example}{例}[chapter]
%  \newtheorem{proof}{証明}
% 証明には番号をつけず,最後は Box で終わります.

% 英語で,見出しのフォントが気に入らなかったら
%\renewcommand{\headfont}{\bfseries}

% ページ数が少ないときはここを大きくしてごまかそう!!効果絶大!!
\renewcommand{\baselinestretch}{1.0}

\begin{document}
\begin{acknowledgement}
\par
本研究を行うにあたり,懇切丁寧なご指導,ご教授を賜りました大阪大学大学院基礎工学研究科システム創成専攻 潮 俊光 教授に心より感謝の意を表します.

また,さまざまな面において励ましや助言をいただきました潮研究室の皆様に深く感謝いたします.
\end{acknowledgement}
\include{end}

% 参考文献
% 1. 直接書く
\begin{thebibliography}{99}
\bibitem{bib1} 山本俊行, K. Liu,森川高行, “タクシー配車データのプローブデータとしての活用に関する基礎的分析,” 土木計画学研究・論文集,Vol. 23, No. 4, pp. 863-870, 2006.
\bibitem{bib2} 金月寛彰,服部宏充,“プローブカーデータを利用したタクシードライバーの個人特性の分析とモデル化,” 第29 回人工知能学会全国大会, 演題番号1N4-4, 2015.
\bibitem{bib3} K. Zhao, S. Tarkoma, S. Liu, and H. Vo, “Urban HumaMobility Data Mining: An Overview,” in Proc. 2016 IEEE International Conference on Big Data, 2016.
\bibitem{bib4} K. Zhao, D. Khryashchev, J. Freire, C. Silva, and H. Vo, “Predicting Taxi Demand at High Spatial Resolution: Approaching the Limit of Predictability,” in Proc. 2016 IEEE International Conference on Big Data, 2016.
\bibitem{bib20} L. Matias, J. Gama, M. Ferreira, J. Mendes-Moreira and L. Damas, Predicting Taxi-Passenger Demand using Streaming Data, IEEE Trans. on Intelligent Transportation Systems, Vol. 14, No. 3, pp. 1393-1402, 2013.
\bibitem{bib19} R. Xu, ”Machine Learning for Real-time Demand Forcasting,” PhD thesis, Massachusetts institute of technology, 2015.
\bibitem{bib5} K. T. Seow, N. H. Dang, and D.-H. Lee, “A Collaborativ Multiagent Taxi-Dispatch System,” IEEE Trans. Automation Science and Engineering, Vol. 7, No. 3, pp. 607-616 ,2010.
\bibitem{bib6} M. Qu, H. Zhu, J. Liu, G. Liu, and H. Xiong, “A Cost-Effective Recommender System for Taxi Drivers,” in Proc. 20th International Conference on KDD, pp. 45-54, 2014.
\bibitem{bib7} F. Miao, S. Lin, S. Munir, J. A. Stankovic, H. Huang, D. Zhang, T. He, G. J. Pappas, “Taxi Dispatch with Real-Time Sensing Data in Metropolitan Areas - A Receding Horizon Control Approach,” in Proc. International Conference on Cyber-Physical Systems, pp. 100-109, 2015.
\bibitem{bib8} F. Miao, S. Han S. Lin, and G. J. Pappas, “Robust Tax Dispatch under Model Uncertainties,” in Proc. 54th IEEE Conference on Decision and Control, pp. 2816-2821, 2015.
\bibitem{bib9} F. MIao, S. Han, S. Lin, Q. Wang, J. Stankovic, A. Hendawi, D. Zhang T. He, and G. Pappas, “Data-Driven Robust Taxi Dispatch under Demand Uncertainties,” arXiv:1603.06263, 2016.
\bibitem{bib10} A. Bemporad and M. Morari, “Control of Systems Integratin Logic, Dynamics, and Constraints,” Automatica, Vol .35, No. 3, pp. 407-427, 1999.
\bibitem{bib11} 井村順一,東俊一,増淵泉,ハイブリッドシステムの制御,コロナ社,2014.
\bibitem{bib12} 中野 良平,ニューラル情報処理の基礎数理,数理工学社,2005.
\bibitem{bib13} C. Bishop,元田他監訳,パターン認識と機械学習(上,下),丸善出版,2008.
\bibitem{bib14} 神嶌敏弘,深層学習,近代科学社,2015.
\bibitem{bib15} F. Rosenblatt, ”The Perceptron: a probabilistic model for information storage and organization in the brain,” Psyhological Review, Vol. 65, No. 6, pp. 386-408, 1958.
\bibitem{bib16} G. Cybenko, ”Approximation by superpositions of a sigmoidal function,” Mathmatics of Control, Signals, and Systems, Vol. 2, No. 4, pp. 303-314, 1989.
\bibitem{bib17} D. E. Rumelhart, G. E. Hinton, and R. J. Williams, ”Learning representations by back-propagating errors,” Nature, Vol. 323, No. 9, pp. 533-536, 1986.
\bibitem{bib18} 大塚敏之他,実時間最適化による制御の実応用,コロナ社,2015.
\end{thebibliography}

% % 2. bibtexを使う
% \bibliography{myjunsrt}
% \bibliography{refs}

% 付録
%卒業論文用雛形
\documentclass[a4j,12pt,oneside,openany]{jsbook}
% 英語なら以下を使う.
%\documentclass[a4j,12pt,oneside,openany,english]{jsbook}

\usepackage[dvipdfmx]{graphicx}
\usepackage[dvipdfmx]{hyperref}
\usepackage{pxjahyper}

\usepackage{amssymb}
\usepackage{amsmath}
\usepackage{latexsym}
\usepackage{overcite}


%jsbook を report っぽくするスタイルファイル
\usepackage{book2report}
%定理,補題,系,例題,証明などや英語用の定義がされています.
%自分なりにいじってください.
\usepackage{thesis}
% 具体的には以下のように定義されています.
% 英語の定理環境
%  \newtheorem{theorem}{Theorem}[chapter]
%  \newtheorem{lemma}{Lemma}[chapter]
%  \newtheorem{proposition}{Proposition}[chapter]
%  \newtheorem{corollary}{Corollary}[chapter]
%  \newtheorem{definition}{Definition}[chapter]
%  \newtheorem{example}{Example}[chapter]
%  \newtheorem{proof}{Proof}
% 日本語の定理環境
%  \newtheorem{theorem}{定理}[chapter]
%  \newtheorem{lemma}{補題}[chapter]
%  \newtheorem{proposition}{命題}[chapter]
%  \newtheorem{corollary}{系}[chapter]
%  \newtheorem{definition}{定義}[chapter]
%  \newtheorem{example}{例}[chapter]
%  \newtheorem{proof}{証明}
% 証明には番号をつけず,最後は Box で終わります.

% 英語で,見出しのフォントが気に入らなかったら
%\renewcommand{\headfont}{\bfseries}

% ページ数が少ないときはここを大きくしてごまかそう!!効果絶大!!
\renewcommand{\baselinestretch}{1.0}

\begin{document}
\appendix
\chapter{混合論理動的システム}
\par
 混合論理動的システム(以下MLDシステムと呼ぶ)モデルは
\begin{align}
 \begin{cases}
  x(k+1) = Ax(k)+B_1u(k)+B_2z(k)+B_3\delta(k)\\
  Cx(k)+D_1u(k)+D_2z(k)+D_3\delta(k)\leq D_4
 \end{cases}
\end{align}
で与えられる.
ここで,$x(k)\in\mathbb{R}^n$は状態,$u(k)\in\mathbb{R}^m$は入力,$z(k)\in\mathbb{R}^{l_1}$と$\delta(k)\in\{0,\ 1\}^{l_2}$は補助変数である.
また,$A\in\mathbb{R}^{n\times n}$,$B_1\in\mathbb{R}^{n\times m}$,$B_2\in\mathbb{R}^{n\times l_1}$,$B_3\in\mathbb{R}^{n\times l_2}$,$C\in\mathbb{R}^{q\times n}$,$D_1\in\mathbb{R}^{q\times m}$,$D_2\in\mathbb{R}^{q\times l_1}$,$D_3\in\mathbb{R}^{q\times l_2}$,$D_4\in\mathbb{R}^{q}$は定数行列である.
補助変数$\delta$は,このモデルの離散状態を表している.

\par
バイナリ変数と論理積,論理和,否定などの論理演算を含む命題論理は,バイナリ変数と四則演算からなる線形不等式で表現できる.
例えば,命題$i$の真偽を表す変数を論理変数と呼び,$X_i\in\{0,\ 1\}$で表す.
そして,$X_i$を命題「$\delta_i=1$である」と対応付け,$X_i=[\delta_i=1]$と表現することにする.
すると,各論理演算について以下の補題が成り立つ.
\lemma{}
\ 
\begin{enumerate}
 \item $[\delta_1=1] \lor [\delta_2=1]\ (=X_1 \lor X_2)$と$\delta_1+\delta_2\geq 1$は等価である.
 \item $[\delta_1=1] \land [\delta_2=1]\ (=X_1 \land X_2)$と$\delta_1=1$,$\delta_2=1$は等価である.
 \item $[\delta_1=1] \to [\delta_2=1]\ (=X_1 \to X_2)$と$\delta_1-\delta_2\leq 0$は等価である.
 \item $[\delta_1=1] \leftrightarrow [\delta_2=1]\ (=X_1 \leftrightarrow X_2)$と$\delta_1-\delta_2 = 0$は等価である.
 \item $[\delta_1=1] \oplus [\delta_2=1]\ (=X_1 \oplus X_2)$と$\delta_1+\delta_2 = 1$は等価である.
\end{enumerate}

最後に本論文で利用した,連続値変数を含む場合の論理条件の不等式表現について補題を示す.
\lemma{}
$\delta\in\{0,\ 1\}$をインデックス変数,$x\in\mathbb{R}^n$を連続値変数とする.
このとき,関数$h\ :\ \mathbb{R}^{n}\to\mathbb{R}$,$g\ :\ \mathbb{R}^{n}\to\mathbb{R}^{m}$に対して,有界集合$\mathbb{X}\subset\mathbb{R}^{n}$上で次の関係が成り立つ.
\begin{enumerate}
 \item $[\delta_1=1] \leftrightarrow [h(x)\geq 0]$は,つぎの線形不等式によって任意の精度で近似できる.
\begin{align}
  h_{\inf}(1-\delta)\leq h(x)\leq h_{\sup}(\delta-1)\epsilon
\end{align}
ただし,$h_{\inf}$,$h_{\sup}\in\mathbb{R}$はすべての$x\in\mathbb{X}$に対して$h_{\inf}\leq h(x)\leq h_{\sup}$を満たすものであり,$\epsilon\in\mathbb{R}_{++}$は任意に選ばれた十分小さい定数である.
 \item $z=\delta g(x)$はつぎの不等式と等価である.
\begin{align}
  g_{\inf}\delta\leq z\leq g_{\sup}\delta
\end{align}
\begin{align}
  g(x)-g_{\sup}(1-\delta)\leq z\leq g(x)-g_{\inf}(1-\delta)
\end{align}
\end{enumerate}
ただし,$g_{\inf}$,$g_{\sup}\in\mathbb{R}^{m}$はすべての$x\in\mathbb{X}$に対して$g_{\inf}\leq g(x)\leq g_{\sup}$を満たすベクトルである.
\chapter{モデル予測制御}
\par
モデル予測制御は現時点から有限区間内の制約条件がシステムのダイナミクスとなっている数理計画問題を解き,得られた有限区間の入力のうち初期入力の1ステップ分のみを制御入力として利用し,各時刻でこれを繰り返し行い制御する方法である.
モデル予測制御は現時点から$T$ステップ先のシステムの状態量を数理計画問題を解くことで予測するため,有限区間の終端時刻$T$は予測ステップ数と呼ばれる.

\par
モデル予測制御には計算時間に関する課題がある.
まず,モデル予測制御を実装するには各時刻でのシステムの状態を計測したら即座に最適制御問題を解いて制御入力を決定する必要がある.
したがって,最適制御問題の数値解法は時刻の時間単位内に終わる必要がある.
\include{end}
% ニューラルネットワーク
% 混合論理動的システム

% 研究業績
%卒業論文用雛形
\documentclass[a4j,12pt,oneside,openany]{jsbook}
% 英語なら以下を使う.
%\documentclass[a4j,12pt,oneside,openany,english]{jsbook}

\usepackage[dvipdfmx]{graphicx}
\usepackage[dvipdfmx]{hyperref}
\usepackage{pxjahyper}

\usepackage{amssymb}
\usepackage{amsmath}
\usepackage{latexsym}
\usepackage{overcite}


%jsbook を report っぽくするスタイルファイル
\usepackage{book2report}
%定理,補題,系,例題,証明などや英語用の定義がされています.
%自分なりにいじってください.
\usepackage{thesis}
% 具体的には以下のように定義されています.
% 英語の定理環境
%  \newtheorem{theorem}{Theorem}[chapter]
%  \newtheorem{lemma}{Lemma}[chapter]
%  \newtheorem{proposition}{Proposition}[chapter]
%  \newtheorem{corollary}{Corollary}[chapter]
%  \newtheorem{definition}{Definition}[chapter]
%  \newtheorem{example}{Example}[chapter]
%  \newtheorem{proof}{Proof}
% 日本語の定理環境
%  \newtheorem{theorem}{定理}[chapter]
%  \newtheorem{lemma}{補題}[chapter]
%  \newtheorem{proposition}{命題}[chapter]
%  \newtheorem{corollary}{系}[chapter]
%  \newtheorem{definition}{定義}[chapter]
%  \newtheorem{example}{例}[chapter]
%  \newtheorem{proof}{証明}
% 証明には番号をつけず,最後は Box で終わります.

% 英語で,見出しのフォントが気に入らなかったら
%\renewcommand{\headfont}{\bfseries}

% ページ数が少ないときはここを大きくしてごまかそう!!効果絶大!!
\renewcommand{\baselinestretch}{1.0}

\begin{document}
\chapter*{研究業績}
\section*{国際会議}
\begin{itemize}
\item Masaki Hiromoto and Toshimitsu Ushio, ``Learning an Optimal Control Policy for a Markov Decision Process Under Linear Temporal Logic Specifications,''in Proceedings of 2015 IEEE Symposium on Adaptive Dynamic Programming and Reinforcement Learning (in 2015 IEEE Symposium Series on Computational Intelligence), Cape Town, South Africa, pp. 548-555, Dec. 2015.
\item Toshimitsu Ushio, Masaki Hiromoto, Akiyoshi Okamoto, and Tomoaki Akiyama, ``WIP Abstract: A Mixed Logical Dynamical System Model for Taxi Cruising Support System,'' In Proceedings of 2016 ACM/IEEE 7th International Conference on Cyber-Physical Systems Vienna, Austria April 2016.
\end{itemize}
\section*{国内会議}
\begin{itemize}
\item 広本 将基,潮 俊光, 「 LTL制約の下でのMDPに対するスーパバイザの強化学習」,2016年電子情報通信学会総合大会,p. 166, 2016.
\item 広本 将基,潮 俊光,岡本 明義,秋山 友昭「モデル予測制御によるタクシーの最適配車問題の定式化」電子情報通信学会技術報告書,MSS2016-76, SS2016-55, pp. 113-116, 2017.
\end{itemize}
\include{end}

\input{end}

