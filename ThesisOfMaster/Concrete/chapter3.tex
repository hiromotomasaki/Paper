%卒業論文用雛形
\documentclass[a4j,12pt,oneside,openany]{jsbook}
% 英語なら以下を使う.
%\documentclass[a4j,12pt,oneside,openany,english]{jsbook}

\usepackage[dvipdfmx]{graphicx}
\usepackage[dvipdfmx]{hyperref}
\usepackage{pxjahyper}

\usepackage{amssymb}
\usepackage{amsmath}
\usepackage{latexsym}
\usepackage{overcite}


%jsbook を report っぽくするスタイルファイル
\usepackage{book2report}
%定理,補題,系,例題,証明などや英語用の定義がされています.
%自分なりにいじってください.
\usepackage{thesis}
% 具体的には以下のように定義されています.
% 英語の定理環境
%  \newtheorem{theorem}{Theorem}[chapter]
%  \newtheorem{lemma}{Lemma}[chapter]
%  \newtheorem{proposition}{Proposition}[chapter]
%  \newtheorem{corollary}{Corollary}[chapter]
%  \newtheorem{definition}{Definition}[chapter]
%  \newtheorem{example}{Example}[chapter]
%  \newtheorem{proof}{Proof}
% 日本語の定理環境
%  \newtheorem{theorem}{定理}[chapter]
%  \newtheorem{lemma}{補題}[chapter]
%  \newtheorem{proposition}{命題}[chapter]
%  \newtheorem{corollary}{系}[chapter]
%  \newtheorem{definition}{定義}[chapter]
%  \newtheorem{example}{例}[chapter]
%  \newtheorem{proof}{証明}
% 証明には番号をつけず,最後は Box で終わります.

% 英語で,見出しのフォントが気に入らなかったら
%\renewcommand{\headfont}{\bfseries}

% ページ数が少ないときはここを大きくしてごまかそう!!効果絶大!!
\renewcommand{\baselinestretch}{1.0}

\begin{document}
\chapter{乗降車データに基づいた最適配車問題の定式化}
\label{ch:3}
 \section{緒言}
 \label{sec:3_1}
 \par
 本章では,タクシーが営業を行う領域をいくつかの部分領域(以下セルと呼ぶ)に分割し,セル間でのタクシーの移動モデルを混合論理動的システムでモデル化する.
 そして,そのモデルを用いたモデル予測制御法を提案し,アプリケーションへの実装結果を示す.
 \section{タクシー移動モデル}
 \label{sec:3_2}
 \par
 対象領域を$N$個のセルに分割する.
 時刻$k$のときのセル$i$($i = 1, 2, \ldots, N$)での空車数を$x_i(k)$とおき,対象領域内のタクシーの実車数を$r(k)$とおく.
 時刻$k$でのセル$i$で空車が実車に変化するタクシー数を$s_i(k)$とおく.
 すなわち,時間区間$[k,\ k+1)$の間に,セル$i$で空車から実車になるタクシー数が$s_i(k)$である.
 時刻$k$でのセル$i$で実車が空車に変化するタクシー数を$e_i(k)$とおく.
さらに入力として,時刻$k$でのセル$i$からセル$j$へ移動する空車数を$u_{i, j}(k)$とおく.
つまり,$x_i(k)$と$r(k)$が時刻$k$におけるシステムの状態量を表し,$s_i(k)$と$e_i(k)$と$u_{i, j}(k)$が時刻$k+1$の状態量を記述するための変動量を表している.このとき,各セルの空車数のダイナミクスは
\begin{align}
 x_i(k+1) = x_i(k)-s_i(k)+e_i(k)+\sum_{j=1}^{N}\bigg(u_{j,i}(k)-u_{i,j}(k) \bigg) \label{eq:x}
\end{align}
となる.また,実車数のダイナミクスは
\begin{align}
 r(k+1) = r(k)+\sum_{i=1}^{N}\bigg(s_i(k)-e_i(k)\bigg) \label{eq:r}
\end{align}
となる.

ここで,式(\ref{eq:x}),(\ref{eq:r})から
\begin{align*}
 &r(k+1)+\sum_{i=1}^{N}x_i(k+1)\nonumber \\
 =\ & \Bigg(r(k)+\sum_{i=1}^{N}\bigg(s_i(k)-e_i(k)\bigg)\Bigg) + \sum_{i=1}^{N}\Bigg(x_i(k)-s_i(k)+e_i(k)+\sum_{j=1}^{N}\bigg(u_{j,i}(k)-u_{i,j}(k) \bigg)\Bigg)\\
 =\ & r(k)+\sum_{i=1}^{N}x_i(k)+\sum_{i=1}^{N}\sum_{j=1}^{N}\bigg(u_{j,i}(k)-u_{i,j}(k) \bigg)\\
 =\ & r(k)+\sum_{i=1}^{N}x_i(k)+\sum_{i=1}^{N}\sum_{j=1}^{N}\bigg(u_{i,j}(k)-u_{i,j}(k) \bigg)\\
 =\ & r(k)+\sum_{i=1}^{N}x_i(k)
\end{align*}
が示せる.
すなわち,タクシーの総数は変化しない.
タクシーの総数を$L$とおくと
\begin{align}
 r(k)= L-\sum_{i=1}^{N}x_i(k) \label{eq:r_new}
\end{align}
の関係が成り立つ.

\par
実車に変化するタクシー数$s_i(k)$については以下のように考える.
まず,時刻$k$での制御入力に従って移動してから実車になりうるとする.
このとき,時間区間$[k,\ k+1)$の間で実車から空車になるタクシーもすぐに実車になりうるので,時間区間$[k,\ k+1)$の間にセル$i$にいる実車になりうるタクシーの台数は$e_i(k)+\sum_{i=1}^{N}u_{j,i}(k)$であり,実車になる台数はこの数を超えることはないので,
\begin{align}
 h_i(k)=e_i(k) +\sum_{j=1}^{N}u_{j,i}(k) -\alpha_i d_i(k) \label{eq:h}
\end{align}
とおくと,
\begin{align}
 s_i(k)=\left\{
\begin{array}{ll}
 \alpha_i d_i(k) & \mbox{if }h_i(k) \geq 0 \\
h_i(k)+\alpha_i d_i(k) & \mbox{otherwise}
\end{array}\right. \label{eq:s}
\end{align}
と表される.
ただし,$\alpha_i$はセル$i$においてタクシーに乗車できる乗客の割合である.
つまり,領域内に乗客と空車のタクシーがいる状況でも,乗客を見つけることが出来ず,実車に変化できない場合を考慮したモデルになっている.
この定数$\alpha_i$は,過去のデータから推定することができる.
$d_i(k)$は時間区間$[k,\ k+1)$の間にセル$i$で乗車できなかった客数であり,
\begin{align}
 d_i(k+1)=d_i(k)-s_i(k)+p_i(k) \label{eq:d}
\end{align}
と表される.
ただし,$p_i(k)$は時間区間$[k,\ k+1)$の間で発生する新たな乗客数で,過去の乗車データから予測される.

\par
空車に変化するタクシー数$e_i(k)$は
\begin{align}
 e_i(k)=\beta_{i}r(k) \label{eq:e}
\end{align}
とする.
ただし,$\beta_i$は実車全体の中でセル$i$で空車になる割合である.
この定数$\beta_i$は,過去のデータから推定することができる.

\par
ここで,入力に関する制約として,各セル$i$について
\begin{align}
 x_i(k)=\sum_{j=1}^{N}u_{i,j}(k) \label{eq:input}
\end{align}
を与える.
このことは,各セルにおいて時刻$k$での空車をどこに移動させるかを決定し,それに沿って,空車が移動すると仮定していることになる.
空車は制御入力に従って移動してから実車に変化できると仮定する.
このように移動する空車数を定めると,式(\ref{eq:x}),(\ref{eq:input})から,各空車数のダイナミクスは
\begin{align}
 x_i(k+1) = e_i(k)-s_i(k)+\sum_{j=1}^{N} u_{j,i}(k) \label{eq:x_new}
\end{align}
となる.
さらに,自動車の移動速度の制約から必ず0になる$u_{i, j}(k)$がある.
例えば,時間単位で隣接するセルにしか移動できない場合には,セル$i$に隣接しないセル$\ell$については
\begin{align}
 u_{i, \ell}(k)=0 \qquad  \forall k \label{eq:ell}
\end{align}
とおく.

\par
以上より,タクシー移動モデルは以下のようになる.
\begin{align}
 x_i(k+1) &= e_i(k)-s_i(k)+\sum_{j=1}^{N} u_{j,i}(k) \label{eq:x1}\\
 r(k) &= L-\sum_{i=1}^{N}x_i(k) \label{eq:r1}\\
 e_i(k) &= \beta_{i}r(k) \label{eq:e1}\\
 s_i(k) &= \left\{
\begin{array}{ll}
 \alpha_i d_i(k) & \mbox{if }h_i(k) \geq 0 \\
h_i(k)+\alpha_i d_i(k) & \mbox{otherwise}
\end{array}\right. \label{eq:s1}\\
 d_i(k+1) &= d_i(k)-s_i(k)+p_i(k) \label{eq:d1}\\
 h_i(k) &= e_i(k) +\sum_{j=1}^{N}u_{j,i}(k) -\alpha_i d_i(k) \label{eq:h1}\\
 x_i(k) &= \sum_{j=1}^{N}u_{i,j}(k) \label{eq:input1}\\
 u_{i, \ell}(k) &= 0 \qquad  \mbox{if セル$i$からセル$j$に移動不可能}\label{eq:ell1}
\end{align}
ここで,式(\ref{eq:s1})が条件付きの式になっている.この式は以下のように変形すれは,システム全体は混合論理動的システムになる\cite{bib10, bib11}.

まず,以下の論理変数$\delta_i(k)\in\{ 0,\ 1\}$を導入する.
\begin{align}
 \delta_i(k)=
\left\{ \begin{array}{ll}
1 & \mbox{if }h_i(k)\geq 0 \\
0 & \mbox{otherwise}
\end{array} \right. \label{eq:delta}
\end{align}
と定義する.
このとき,制約条件式(\ref{eq:delta})は次の不等式制約条件になる\cite{bib10}.
\begin{align}
 h^{\inf}_{i}(k)(1-\delta_i(k))\leq h_i(k) \leq h^{\sup}_{i}(k) \delta_i(k)+(\delta_i(k) -1) \epsilon_i(k) \label{eq:delta_new}
\end{align}
ただし,$h^{\inf}_{i}(k)$,$h^{\sup}_{i}(k)\in\mathbb{R}$は$h_i(k)$の引数が取りうる任意の値に対して$h^{\inf}_{i}(k)\leq h_i(k)\leq h^{\sup}_{i}(k)$であり,$\epsilon_i(k)\in\mathbb{R}_{++}$は十分に小さな正の実数である.
実際に,式(\ref{eq:delta_new})は$\delta_i(k)=1$のときは
\begin{align*}
 0\leq h_i(k) \leq h^{\sup}_{i}(k)
\end{align*}
となり,$\delta_i(k)=0$のときは
\begin{align*}
 h^{\inf}_{i}(k)\leq h_i(k) \leq -\epsilon_i(k)\ (<0)
\end{align*}
となるので,$\epsilon_i(k)$の値を十分に小さくすれば,任意の精度で制約条件式を不等式制約式に変換可能であることが確認できる.
また,初期時刻を$k=t$とおくと,$k>t$の$h_i(k)$について式(\ref{eq:d1})から以下の不等式が成り立つ.
\begin{align*}
 h_{i}(k) &= e_i(k) +\sum_{j=1}^{N}u_{j,i}(k) -\alpha_i d_i(k)\\
&\geq -\alpha_i d_i(k)\\
&= -\alpha_i \bigg( d_i(k-1)-s_i(k-1)+p_i(k-1) \bigg)\\
&\geq -\alpha_i \bigg( d_i(k-1)+p_i(k-1) \bigg)\\
&\geq -\alpha_i \bigg( d_i(k-2)+p_i(k-2)+p_i(k-1) \bigg)\\
&\geq \cdots\\
&\geq -\alpha_i \bigg( d_i(t)+\sum_{c=t}^{k-1}p_i(c) \bigg)\\
 h_{i}(k) &\leq e_i(k) +\sum_{j=1}^{N}u_{j,i}(k)\\
&\leq L\\
\end{align*}
したがって,$h_i(k)$の上界と下界を以下のように定める.
\begin{align}
 h_{i}^{\sup}(k) &= L\label{eq:h_sup}\\
 h_{i}^{\inf}(k) &= -\alpha_i \bigg( d_i(t)+\sum_{c=t}^{k-1}p_i(c) \bigg)\label{eq:h_inf}
\end{align}

論理変数$\delta_i(k)$を用いることで式(\ref{eq:s1})は以下のように変形できる.
\begin{align}
s_i(k) &= \delta_i(k)\alpha_id_i(k)+(1-\delta_i(k))(h_i(k)+\alpha_i d_i(k))\nonumber\\
&= -\delta_i(k)h_i(k)+h_i(k)+\alpha_i d_i(k)\label{eq:s1_new}
\end{align}
ここで,
\begin{align}
z_i(k) = \delta_i(k)h_i(k)\label{eq:z1}
\end{align}
とおくと,式(\ref{eq:s1_new})は
\begin{align}
s_i(k) = -z_i(k)+h_i(k)+\alpha_i d_i(k)\label{eq:s1_newnew}
\end{align}
となる.
式(\ref{eq:z1})は次の不等式制約条件になる\cite{bib11}.
\begin{align}
 & h^{\inf}_{i}(k) \delta_i(k) \leq z_i(k) \leq h^{\sup}_{i}(k) \delta_i(k)\label{eq:z1_1}\\
& h_i(k)-h^{\sup}_i(k) (1-\delta_i(k)) \leq z_i(k) \leq h_i(k)-h^{\inf}_i(k) (1-\delta_i(k))\label{eq:z1_2}
\end{align}

\par
以上より,初期時刻を$k=t$とおくと,タクシー移動モデルは以下の混合論理動的システムで記述される.
\begin{align}
 & x_i(k+1) = e_i(k)-s_i(k)+\sum_{j=1}^{N} u_{j,i}(k) \label{eq:x2}\\
 & r(k) = L-\sum_{i=1}^{N}x_i(k) \label{eq:r2}\\
 & e_i(k) = \beta_{i}r(k) \label{eq:e2}\\
 & h^{\inf}_{i}(k)(1-\delta_i(k))\leq h_i(k) \leq h^{\sup}_{i}(k) \delta_i(k)+(\delta_i(k) -1) \epsilon_i(k) \label{eq:delta2}\\
 & s_i(k) = -z_i(k)+h_i(k)+\alpha_i d_i(k)\label{eq:s2}\\
 & h^{\inf}_{i}(k) \delta_i(k) \leq z_i(k) \leq h^{\sup}_{i}(k) \delta_i(k)\label{eq:z2_1}\\
 & h_i(k)-h^{\sup}_i(k) (1-\delta_i(k)) \leq z_i(k) \leq h_i(k)-h^{\inf}_i(k) (1-\delta_i(k))\label{eq:z2_2}\\
 & h_{i}^{\sup}(k) = L\label{eq:h2_sup}\\
 & h_{i}^{\inf}(k) = -\alpha_i \bigg( d_i(t)+\sum_{c=t}^{k-1}p_i(c) \bigg)\label{eq:h2_inf}\\
 & d_i(k+1) = d_i(k)-s_i(k)+p_i(k) \label{eq:d2}\\
 & h_i(k) = e_i(k) +\sum_{j=1}^{N}u_{j,i}(k) -\alpha_i d_i(k) \label{eq:h2}\\
 & x_i(k) = \sum_{j=1}^{N}u_{i,j}(k) \label{eq:input2}\\
 & u_{i, \ell}(k) = 0 \qquad  \mbox{if セル$i$からセル$j$に移動不可能}\label{eq:ell2}
\end{align}
 \section{モデル予測制御}
 \label{sec:3_3}
 \subsection{定式化}
 \label{sec:3_3_1}
 節\ref{sec:3_2}で導出したタクシー移動モデルをもとに,時刻$t$において以下の有限区間最適制御問題を考える.
ただし,$T$は正の整数である.
  \begin{align}
 \mbox{minimize }& J_t=\sum_{i=1}^{N}d_i(t+T)\label{eq:J_t}\\
  \mbox{制約条件 : }&\mbox{各$i=1, 2, \ldots, N$と$k=t, t+1, \ldots, t+T-1$について}\nonumber\\
 & x_i(k+1) = e_i(k)-s_i(k)+\sum_{j=1}^{N} u_{j,i}(k) \label{eq:x3}\\
 & r(k) = L-\sum_{i=1}^{N}x_i(k) \label{eq:r3}\\
 & e_i(k) = \beta_{i}r(k) \label{eq:e3}\\
 & h^{\inf}_{i}(k)(1-\delta_i(k))\leq h_i(k) \leq h^{\sup}_{i}(k) \delta_i(k)+(\delta_i(k) -1) \epsilon_i(k) \label{eq:delta3}\\
 & s_i(k) = -z_i(k)+h_i(k)+\alpha_i d_i(k)\label{eq:s3}\\
 & h^{\inf}_{i}(k) \delta_i(k) \leq z_i(k) \leq h^{\sup}_{i}(k) \delta_i(k)\label{eq:z3_1}\\
 & h_i(k)-h^{\sup}_i(k) (1-\delta_i(k)) \leq z_i(k) \leq h_i(k)-h^{\inf}_i(k) (1-\delta_i(k))\label{eq:z3_2}\\
 & h_{i}^{\sup}(k) = L\label{eq:h3_sup}\\
 & h_{i}^{\inf}(k) = -\alpha_i \bigg( d_i(t)+\sum_{c=t}^{k-1}p_i(c) \bigg)\label{eq:h3_inf}\\
 & d_i(k+1) = d_i(k)-s_i(k)+p_i(k) \label{eq:d3}\\
 & h_i(k) = e_i(k) +\sum_{j=1}^{N}u_{j,i}(k) -\alpha_i d_i(k) \label{eq:h3}\\
 & x_i(k) = \sum_{j=1}^{N}u_{i,j}(k) \label{eq:input3}\\
 & u_{i, \ell}(k) = 0 \qquad  \mbox{if セル$i$からセル$j$に移動不可能}\label{eq:ell3}\\
 & x_i(t) = \mbox{given}\\
 & d_i(t) = \mbox{given}
\end{align}
自明なものも含めると,変数の数は$N^2T+6NT+2N+T$,制約条件の数は$10NT+2N$である.
制約条件は線形の式であり,変数が実数と整数のどちらもあるので,求める最適化問題は混合整数計画問題として定式化できる.
コスト関数(\ref{eq:J_t})は各時刻$t$において,時刻$t+T$での乗車できない乗客数の総和を最小にすることを意味する.

  \subsection{実装結果}
  \label{sec:3_3_2}
アプリケーション上に実装した結果を示す.
対象領域は大阪駅から難波あたりまでとし,一辺の長さ2kmの正方形のセルを使って,東西に6セル,南北に4セルに分割した(N=24).
過去の乗降車データからは乗車率$\alpha_i$を推定できないため,すべてのセルにおいて乗車率$\alpha_i=0.9$とした.
2分を1時間単位として,2時間単位区間(T=2)でモデル予測制御により最適移動分布を計算した.
\begin{figure}[t]
  \begin{minipage}[b]{0.24\linewidth}
    \centering
    \includegraphics[keepaspectratio, width=22mm]{Graphics/chapter3/201603311802.jpg}
    \subcaption{18時2分}\label{fig:fig3_3_2_1}
  \end{minipage}
  \begin{minipage}[b]{0.24\linewidth}
    \centering
    \includegraphics[keepaspectratio, width=22mm]{Graphics/chapter3/201603311809.jpg}
    \subcaption{18時9分}\label{fig:fig3_3_2_2}
  \end{minipage}
  \begin{minipage}[b]{0.24\linewidth}
    \centering
    \includegraphics[keepaspectratio, width=22mm]{Graphics/chapter3/201603311815.jpg}
    \subcaption{18時15分}\label{fig:fig3_3_2_3}
  \end{minipage}
  \begin{minipage}[b]{0.24\linewidth}
    \centering
    \includegraphics[keepaspectratio, width=22mm]{Graphics/chapter3/201603311824.jpg}
    \subcaption{18時24分}\label{fig:fig3_3_2_4}
  \end{minipage}
  \caption{最適方向の時間的変化(2016年3月31日)}\label{fig:fig3_3_2}
\end{figure}
図\ref{fig:fig3_3_2}に運転手に提示する画面の時間変化を示す.
青色の方向が推奨する最適な方向,すなわち,最適化問題を解くことで得られた,移動するタクシー数が最大となる方向である.
ピンは,過去の4週間の同曜日の同時刻から2分前から4分後までに乗客を乗せた場所を表す.
1辺の長さ200mの正方形のセルで分割された対象領域の,3km範囲内のセルの中で最も乗車数が多いセルをを赤色の方向(以下,貪欲な方向と呼ぶ)が示している.
18時2分から15分までは,最適な方向と貪欲な方向が異なっており,18時24分にはほぼ同じ方向を示している.
最多乗車データのある領域セルが必ずしも最適な方向であるとは限らないことがわかる.
これは,最多乗車データのあるセル周辺に多くの空車タクシーがいる場合には,むしろタクシーを分散させて,対象領域全体として乗客の獲得を図るような最適移動分布を求めているからである.

 \section{結言}
 \label{sec:3_4}
 \par
 本章では,流しのタクシーの走行支援を目的とした最適移動分布を求める方法を提案した.
対象領域をいくつかの分割領域(セル)に分割し,各セル間のタクシーの移動を表すマクロモデルを混合論理動的システムを用いて表し,モデル予測制御によって最適な移動分布を求めた.
大阪駅から難波までを対象領域として実データをもとに最適な移動方向分布を計算した結果を示した.
乗客数が多いと予想される方向と最も好ましい移動方向とが必ずしも一致しなかった.
このことから,対象領域全体の予測乗客分布と流しのタクシーの現在の分布をもとにして支援を行うことが重要であることがわかる.

\par
今回の実験では,各セルの一辺の長さを2kmとした.
また,1時間単位に進むことができるセルは隣接するセルのみとした.
長さを短くし,隣接するセル以外にも空車が移動できるようにすれば,より精度よく最適移動方向を計算できるが,計算時間が急速に増大する.
計算の並列化,最適解の近似などによる計算時間の短縮が,今後の研究課題である.

 \include{end}