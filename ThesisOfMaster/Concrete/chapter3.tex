%卒業論文用雛形
\documentclass[a4j,12pt,oneside,openany]{jsbook}
% 英語なら以下を使う.
%\documentclass[a4j,12pt,oneside,openany,english]{jsbook}

\usepackage[dvipdfmx]{graphicx}
\usepackage{amssymb}
\usepackage{amsmath}
\usepackage{latexsym}

%jsbook を report っぽくするスタイルファイル
\usepackage{book2report}
%定理,補題,系,例題,証明などや英語用の定義がされています.
%自分なりにいじってください.
\usepackage{thesis}
% 具体的には以下のように定義されています.
% 英語の定理環境
%  \newtheorem{theorem}{Theorem}[chapter]
%  \newtheorem{lemma}{Lemma}[chapter]
%  \newtheorem{proposition}{Proposition}[chapter]
%  \newtheorem{corollary}{Corollary}[chapter]
%  \newtheorem{definition}{Definition}[chapter]
%  \newtheorem{example}{Example}[chapter]
%  \newtheorem{proof}{Proof}
% 日本語の定理環境
%  \newtheorem{theorem}{定理}[chapter]
%  \newtheorem{lemma}{補題}[chapter]
%  \newtheorem{proposition}{命題}[chapter]
%  \newtheorem{corollary}{系}[chapter]
%  \newtheorem{definition}{定義}[chapter]
%  \newtheorem{example}{例}[chapter]
%  \newtheorem{proof}{証明}
% 証明には番号をつけず,最後は Box で終わります.

% 英語で,見出しのフォントが気に入らなかったら
%\renewcommand{\headfont}{\bfseries}

% ページ数が少ないときはここを大きくしてごまかそう!!効果絶大!!
\renewcommand{\baselinestretch}{1.0}

\begin{document}
\chapter{モデル予測制御 - 集中型最適化の場合 -}
\label{ch:3}
 \section{緒言}
 \label{sec:3_1}
 \par
 あ
 \section{あ}
 \label{sec:3_2}
 \par
時刻$k$のときの各領域(セル)$i(i=1, 2, \ldots , N)$での空車数を$x_i(k)$とおく.時間区間$[k,\ k+1)$の間に,領域$i$で実車になるタクシー数を$s_i(k)$,空車になるタクシー数を$e_i(k)$とおく.入力として,時間区間$[k,\ k+1)$に間に領域$i$から$j$へ移動する空車数$u_{i,j}(k)$とおく.ここで,$u_{i,i}(k)$は領域$i$にとどまる空車数である.すると各領域の空車数のダイナミクスは,
\begin{equation}  \label{eqn:system}
x_i(k+1)=x_i(k)-s_i(k)+e_i(k)+\sum_{j=1}^{N}\bigg (u_{j,i}(k)-u_{i,j}(k) \bigg)
\end{equation}
となる.

$d_i(k)$を時刻$k$のときの領域$i$で発生する需要とおく.
\begin{equation} \label{eqn:d}
d_i(k+1)=d_i(k)-s_i(k)+p_i(k)
\end{equation}
である.ただし,$p_i(k)$は時間区間$[k,\ k+1)$の間で新たに発生する需要であり,実データから予想される.制御理論的に言えば,外乱のようなもの.
時間区間$[k,k+1)$に間に空車になるタクシー数$e_i(k)$は
\begin{equation} \label{eqn:e_i}
e_i(k)=\beta_{i}r(k)
\end{equation}
ただし,$r(k)$は時刻$k$のときの実車の総数で,$\beta_{i}$は領域$i$で実車全体の中で領域$i$で空車になる割合である.つまり,時刻$k$のときの実車の中から$\beta_{i}r(k)$だけが時間区間$[k,\ k+1)$の間で領域$i$で空車になることを表し,$\beta_i$は実データから推定される.
実車の総数$r(k)$は,式(\ref{eqn:e_i})を用いると
\begin{eqnarray}
r(k+1) &=& r(k)+\sum_{i=1}^{N}\bigg( s_i(k)-e_i(k)\bigg) \nonumber \\
	&=& r(k)+\sum_{i=1}^{N}\bigg( s_i(k)-\beta_i r(k)\bigg) \nonumber  \\
	&=& \bigg( 1-\sum_{i=1}^{N}\beta_i\bigg) r(k)+\sum_{i=1}^{N} s_i(k) \label{eqn:r}
\end{eqnarray}
である.ここで,入力に関する制約として,各領域$i$について
\begin{equation} \label{eqn:con}
x_i(k)=\sum_{j=1}^{N}u_{i,j}(k)
\end{equation}
を与える.このことは,各領域において時刻$k$での空車をどこに移動させるかを決定し,それに沿って,空車が移動すると仮定していることになる.空車は制御入力に従って移動してから乗車できると仮定する.

このように入力(移動する空車数))を与えると,式(\ref{eqn:system}), (\ref{eqn:e_i}), (\ref{eqn:con})からシステムダイナミクスは
\begin{equation}  \label{eqn:simple_system}
x_i(k+1)=\beta_i r(k)-s_i(k)+\sum_{j=1}^{N} u_{j,i}(k)
\end{equation}
となる.ここで,式(\ref{eqn:r}), (\ref{eqn:con}), (\ref{eqn:simple_system})から
\begin{eqnarray*}
r(k+1)+\sum_{i=1}^{N}x_i(k+1) &=&  \bigg( 1-\sum_{i=1}^{N}\beta_i\bigg) r(k)+\sum_{i=1}^{N} s_i(k) + \sum_{i=1}^{N}\bigg( \beta_i r(k)-s_i(k)+\sum_{j=1}^{N} u_{j,i}(k) \bigg)\\
&=& r(k)+\sum_{i=1}^{N}x_i(k)
\end{eqnarray*}
この式は,タクシーの総数が時間で変動しないことを意味する.もともとタクシーの総数が変化するような制御をかけていないので,この結果は妥当といえる.
そこで,タクシーの総数を$L$とおくと
\begin{equation}
r(k)= L-\sum_{i=1}^{N}x_i(k)
\end{equation}
とおける.したがって,式(\ref{eqn:simple_system})は
\begin{equation}  \label{eqn:simple_system2}
x_i(k+1)=\beta_i \bigg(L-\sum_{j=1}^{N} x_j(k) \bigg) -s_i(k)+\sum_{j=1}^{N} u_{j,i}(k)
\end{equation}
さらに自動車の移動から必ず$0$にしなければならない$u_{i,j}(k)$
があるはずです.例えば,領域$i$に隣接しない領域$\ell$については
\begin{equation} \label{eqn:u_seiyaku}
u_{i, \ell}(k)=0 \qquad  \forall k
\end{equation}
とおいてもよい.このような入力は,式(\ref{eqn:simple_system})からはずしておいてよいでしょう.

以上より,タクシーの総数$L$を用いて,制約条件は以下のようになる.
\begin{eqnarray}
x_i(k+1) &=& \beta_i \bigg(L-\sum_{j=1}^{N} x_j(k) \bigg) -s_i(k)+\sum_{j=1}^{N} u_{j,i}(k) \\
d_i(k+1) &=& d_i(k)-s_i(k)+p_i(k)
\end{eqnarray}
ここで,時間区間$[k,\ k+1)$の間で実車になる台数$s_i(k)$については以下のように考える.まず,時刻$k$での空車が制御入力に従って,移動してから実車になりうるとする.このとき,時間区間$[k,\ k+1)$の間で実車から空車になるタクシーもすぐに実車になりうるので,時間区間$[k,\ k+1)$の間に領域$i$にいる実車になりうるタクシーの台数は$\beta_i r(k)+\sum_{j=1}^{N}u_{j,i}(k)$であり,実車になる台数はこの数を超えることはないので,
\begin{equation} \label{eqn:s2}
s_i(k)=\left\{
\begin{array}{ll}
 \alpha_i d_i(k) & \mbox{if }\beta_i \bigg( L-\sum_{j=1}^{N}x_j(k)\bigg) +\sum_{j=1}^{N}u_{j,i}(k) -\alpha_i d_i(k) \geq 0 \\
\beta_i \bigg( L-\sum_{j=1}^{N}x_j(k)\bigg) +\sum_{j=1}^{N}u_{j,i}(k) & \mbox{otherwise}
\end{array}\right.
\end{equation}
ただし,$\alpha_i$は領域$i$においてタクシーに乗車できる乗客の割合(定数・同定する必要あり.)である.
ここで,式(\ref{eqn:s2})が条件付きの式になっている.この式は以下のように変形すれは,システム全体はMLD(Mixed Logical Dynamical)システムになる.なお,MLDシステムの性質については松本さんの卒論を見てください.

まず,以下の論理変数$\delta_i(k) \in \{ 0,\ 1\}$を導入する.
\begin{equation} \label{eqn:delta}
\delta_i(k)=
\left\{ \begin{array}{ll}
1 & \mbox{if }h_i(x_1(k),\ \ldots, x_N(k),\ u_{1,i}(k),\ldots, u_{N,i}(k),d_i(k))\geq 0 \\
0 & \mbox{otherwise}
\end{array} \right.
\end{equation}
と定義する.ただし,
\begin{equation} \label{eqn:h}
h_i(x_1,\  \ldots, x_N,\ u_{1,i},\ldots, u_{N,i},\ d_i)=\beta_i \bigg( L-\sum_{j=1}^{N}x_j\bigg) +\sum_{j=1}^{N}u_{j,i}(k) -\alpha_i d_i
\end{equation}
である.このとき,松本さんの卒論の補題2.3(i)から,制約条件式(\ref{eqn:delta})は次の不等式制約条件になる.
\begin{eqnarray} \label{ineq:delta}
h^{inf}_{i}(1-\delta_i(k))\leq h_i(k) \leq h^{sup}_{i} \delta_i(k)+(\delta_i(k) -1) \epsilon
\end{eqnarray}
ただし$h_i(k)=h_i(x_1(k),\  \ldots, x_N(k),\ u_{1,i}(k),\ldots, u_{N,i}(k),\ d_i(k))$と略記し,$h^{inf}_{i}$, $h^{sup}_{i} \in R$は取りうる任意の$(x_1,\  \ldots, x_N,\ u_{1,i},\ldots, u_{N,i},\ d_i)$に対して$h^{inf}_{i}\leq h_i(x_1,\  \ldots, x_N,\ u_{1,i},\ldots, u_{N,i},\ d_i)\leq h^{sup}_{i}$であり,$\epsilon \in R_{+}$は十分に小さな正の実数である(この数は$i$に独立でよいでしょう.,$h^{inf}_{i}$, $h^{sup}_{i}$も$i$に独立に指定してもよいかもしれません.).

式(\ref{eqn:s2})は以下のように変形できる.
\begin{eqnarray}
s_i(k) &=& \delta_i(k) \alpha_i d_i(k)+(1-\delta_i(k))S_i(k) \\
	&=& -\delta_i(k)h_i(k)+S_i(k)
\end{eqnarray}
ただし,
\[
S_i(k)=\beta_i \bigg( L-\sum_{j=1}^{N}x_j(k)\bigg) +\sum_{j=1}^{N}u_{j,i}(k)
\]
である.
ここで,
\begin{equation} \label{eqn:z}
z_i(k)=-\delta_i (k) h_i(k) \qquad \bigg( =\delta_i(k) ( -h_i(k)) \bigg)
\end{equation}
とおくと,
\begin{equation} \label{eqn:s_z_x}
s_i(k)=z_i(k)+S_i(k)
\end{equation}
となる,ここで,松本さんの卒論の補題2.3(ii)から,式(\ref{eqn:z})は,
\begin{eqnarray}
& -h^{sup}_{i} \delta_i(k) \leq z_i(k) \leq -h^{inf}_{i} \delta_i(k) & \label{ineq:z1}\\
& -h_i(k)+h^{inf}_i (1-\delta_i(k)) \leq z_i(k) 
   \leq -h_i(k)+h^{sup}_i (1-\delta_i(k)) & \label{ineq:z2}
\end{eqnarray}
と変換できる.

これらの式は線形なので,求める最適化問題は混合整数計画問題(制約条件式が線形の等式または不等式で書かれている.)として定式化できる.
 \section{結言}
 \label{sec:3_3}
 \par
 あ
 \end{document}