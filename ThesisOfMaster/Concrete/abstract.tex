%卒業論文用雛形
\documentclass[a4j,12pt,oneside,openany]{jsbook}
% 英語なら以下を使う.
%\documentclass[a4j,12pt,oneside,openany,english]{jsbook}

\usepackage[dvipdfmx]{graphicx}
\usepackage{amssymb}
\usepackage{amsmath}
\usepackage{latexsym}

%jsbook を report っぽくするスタイルファイル
\usepackage{book2report}
%定理,補題,系,例題,証明などや英語用の定義がされています.
%自分なりにいじってください.
\usepackage{thesis}
% 具体的には以下のように定義されています.
% 英語の定理環境
%  \newtheorem{theorem}{Theorem}[chapter]
%  \newtheorem{lemma}{Lemma}[chapter]
%  \newtheorem{proposition}{Proposition}[chapter]
%  \newtheorem{corollary}{Corollary}[chapter]
%  \newtheorem{definition}{Definition}[chapter]
%  \newtheorem{example}{Example}[chapter]
%  \newtheorem{proof}{Proof}
% 日本語の定理環境
%  \newtheorem{theorem}{定理}[chapter]
%  \newtheorem{lemma}{補題}[chapter]
%  \newtheorem{proposition}{命題}[chapter]
%  \newtheorem{corollary}{系}[chapter]
%  \newtheorem{definition}{定義}[chapter]
%  \newtheorem{example}{例}[chapter]
%  \newtheorem{proof}{証明}
% 証明には番号をつけず,最後は Box で終わります.

% 英語で,見出しのフォントが気に入らなかったら
%\renewcommand{\headfont}{\bfseries}

% ページ数が少ないときはここを大きくしてごまかそう!!効果絶大!!
\renewcommand{\baselinestretch}{1.0}

\begin{document}
\begin{abstract}

\par
 流しのタクシーが効率よく乗客を乗せるための運転支援システムの開発は,運転手の待遇改善につながる重要な課題である.
 本論文では,過去の乗車データと現在の流しのタクシーの分布から最適な進行方向を決定する方法を提案する.
 対象領域をいくつかの分割領域に分割し,各部分領域での需要予測をもとに部分領域ごとの流しのタクシーの変化を混合論理動的システムを使ってモデル化する.
 そして,モデル予測制御を応用して,各部分領域でのタクシーの最適移動分布を求める.
 実際に実装したソフトウェアを示す.




 % 提案手法の有効性は,流しのタクシーが周囲の需要に対して貪欲に運行した場合との比較を行うことによって示す.

%  本論文ではタクシー乗務員の運行をサポートするシステムと,合理的な運行のための制御器を提案する.
% また,その制御器の有効性を個々のドライバーが貪欲に運行した場合と比較を行うことによって示す.


% \par
%  流しのタクシーが効率よく乗客を乗せるための運転支援システムの開発は,運転手の待遇改善につながる重要な課題である.
% また,内閣府は自動走行システムの開発・実用化等を推進する方針を示しており,名古屋ではタクシーの自動運転による実証実験が行われている.
% こうした状況では,データに基づく配車や運行の方法を考えることは重要である.

% また,名古屋ではタクシーの自動運転による実証実験が行われている.
%  内閣府も、自動走行システムの開発・実用化等を推進する方針を示
%  また,名古屋ではタクシーの自動運転による実証実験が行われている.
%  こうした状況では,データに基づく配車や運行の方法を考えることは重要である.

%  \par
%  一方,近年では通信環境が整備され,プロセッサーの性能が向上し,通信用チップが安価に入手できるようになった.
%  つまり,大量のデータを観測,収集し,解析することが容易になった.
%  そのため,サイバーフィジカルシステムの考え方に基づく制御が注目を浴びている.

%  \par
%  本報告では,過去の乗車データと現在の流しのタクシーの分布から最適な進行方向を決定する方法を提案する.
%  対象領域をいくつかの分割領域に分割し,各部分領域での需要予測をもとに部分領域ごとの流しのタクシーの変化を混合論理ダイナミカルシステムを使ってモデル化する.
%  モデル予測制御


% タクシー運転手の労働環境改善のためには,効率よく乗客を乗せることが
% つまり,タクシーの空車時間や空車走行距離を減らすことが改善項目となる.
% 本報告では,過去の乗車
%  \par
%  タクシー業界は道路運送法の下で様々な規制がかけられていた.
%  しかし,2002年に道路運送法が改正され,規制緩和が行われた.
%  そのため,タクシー会社の新規参入が増え,都市部でのタクシーの供給が増えた.
%  また,名古屋ではタクシーの自動運転による実証実験が行われている.
%  こうした状況では,データに基づく配車や運行の方法を考えることは重要である.
%  \par
%  一方,近年では通信環境が整備され,プロセッサーの性能が向上し,通信用チップが安価に入手できるようになった.
%  つまり,大量のデータを観測,収集し,解析することが容易になった.
%  そのため,サイバーフィジカルシステムの考え方に基づく制御が注目を浴びている.
%  \par
%  本論文ではタクシー乗務員の運行をサポートするシステムと,合理的な運行のための制御器を提案する.
% また,その制御器の有効性を個々のドライバーが貪欲に運行した場合と比較を行うことによって示す.
\end{abstract}
\end{document}