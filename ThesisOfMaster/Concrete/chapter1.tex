%卒業論文用雛形
\documentclass[a4j,12pt,oneside,openany]{jsbook}
% 英語なら以下を使う.
%\documentclass[a4j,12pt,oneside,openany,english]{jsbook}

\usepackage[dvipdfmx]{graphicx}
\usepackage[dvipdfmx]{hyperref}
\usepackage{pxjahyper}

\usepackage{amssymb}
\usepackage{amsmath}
\usepackage{latexsym}
\usepackage{overcite}


%jsbook を report っぽくするスタイルファイル
\usepackage{book2report}
%定理,補題,系,例題,証明などや英語用の定義がされています.
%自分なりにいじってください.
\usepackage{thesis}
% 具体的には以下のように定義されています.
% 英語の定理環境
%  \newtheorem{theorem}{Theorem}[chapter]
%  \newtheorem{lemma}{Lemma}[chapter]
%  \newtheorem{proposition}{Proposition}[chapter]
%  \newtheorem{corollary}{Corollary}[chapter]
%  \newtheorem{definition}{Definition}[chapter]
%  \newtheorem{example}{Example}[chapter]
%  \newtheorem{proof}{Proof}
% 日本語の定理環境
%  \newtheorem{theorem}{定理}[chapter]
%  \newtheorem{lemma}{補題}[chapter]
%  \newtheorem{proposition}{命題}[chapter]
%  \newtheorem{corollary}{系}[chapter]
%  \newtheorem{definition}{定義}[chapter]
%  \newtheorem{example}{例}[chapter]
%  \newtheorem{proof}{証明}
% 証明には番号をつけず,最後は Box で終わります.

% 英語で,見出しのフォントが気に入らなかったら
%\renewcommand{\headfont}{\bfseries}

% ページ数が少ないときはここを大きくしてごまかそう!!効果絶大!!
\renewcommand{\baselinestretch}{1.0}

\begin{document}
\chapter{緒論}
\label{ch:1}
 \section{研究背景と目的}
 \label{sec:1_1}

\par
現在のタクシー業界は,高年齢化,低賃金,劣悪な労働環境という問題を抱えている.
流しのタクシーが空車で走行する距離を減らすことで,賃金の向上が達成出来るだけでなく,CO2排出量の削減にもつながる.
現状では,運転手の経験と勘から流し走行をしており,経験の浅い運転手への流し運転の支援は重要な課題である.
最近では,すべてのタクシーにGPSが装着されており,乗客を乗せた位置,一定走行時間・距離ごとの位置情報が無線でリアルタイムに会社へ送信されて,管理できるようになった.
また,名古屋ではタクシーの自動運転による実証実験が行われている.
こうした状況では,ビッグデータを活用して,顧客の発生予測をして,走行方向の支援を行うシステムを考えることは重要である.

\par
一方,インテリジェント交通システムでは,自動車の走行データをオンラインでセンシングできるプローブカーを用いた交通状況のモニタリング法が開発されてきた\cite{bib1}.
タクシーはプローブカートして重要な枠割を果たしているだけでなく,走行履歴から運転手の特性を推定することも可能となってきている\cite{bib2}.
さらに,携帯電話の位置情報や乗車履歴データから乗客の予測技術も急速に発達してきた\cite{bib3, bib4}.

\par
タクシーの運行状況のモニタリング,乗客の予測技術の発展に伴い,最近,タクシーの最適配車の研究が注目されている.
Seow等は,乗客からの呼びに対して乗客の待ち時間が最小となるようなタクシーの配車法を提案している\cite{bib5}.
Qu等は,空車で走行する距離の総和が最小になるような走行方法を提案している\cite{bib6}.
Miao等は,モデル予測制御を用いた最適配車法を提案し,サンフランシスコの市街地を対象に実証実験を行っている\cite{bib7, bib8, bib9}.

\par
本論文では,まず,タクシー乗務員の運行をサポートするシステムを提案する.
そして,タクシーの移動モデルを混合論理動的システムでモデル化し,合理的な運行を行うための制御器を提案する.
提案モデルでは,個々のタクシーに対して個別の制御入力を行うのではなく,部分領域に分けられた中にいるタクシーに対しては同一の制御入力を行う.
その点が文献\cite{bib7, bib8, bib9}とは異なる点である.

 % \par
 % タクシー業界は道路運送法の下で様々な規制がかけられていた.
 % しかし,2002年に道路運送法が改正され,規制緩和が行われた.
 % そのため,タクシー会社の新規参入が増え,都市部でのタクシーの供給が増えた.
 % また,名古屋ではタクシーの自動運転による実証実験が行われている.
 % こうした状況では,データに基づく配車や運行の方法を考えることは重要である.

 % \par
 % 一方,近年では通信環境が整備され,プロセッサーの性能が向上し,通信用チップが安価に入手できるようになった.
 % つまり,大量のデータを観測,収集し,解析することが容易になった.
 % そのため,サイバーフィジカルシステムの考え方に基づく制御が注目を浴びている.

 % \par
 % 本論文ではタクシー乗務員の運行をサポートするシステムと,合理的な運行をするための制御器を提案する.
 % また,その制御器の有効性を個々のドライバーが貪欲に運行した場合と比較を行うことによって示す.

 \section{論文の構成}
 \label{sec:1_2}

 \par
 本報告の構成について述べる.
 第2章では,私達が利用できるデータと提案するシステムについて述べる.
 第3章では, タクシーが営業を行う領域をいくつかの部分領域(セル)に分割し,セル間でのタクシーの移動モデルを混合論理動的システムでモデル化する.
 そして,そのモデルを用いたモデル予測制御法を提案しアプリケーションへの実装結果を示す.
 最後に,第4章では結論と今後の課題について述べる.

 \include{end}
