%卒業論文用雛形
\documentclass[a4j,12pt,oneside,openany]{jsbook}
% 英語なら以下を使う.
%\documentclass[a4j,12pt,oneside,openany,english]{jsbook}

\usepackage[dvipdfmx]{graphicx}
\usepackage[dvipdfmx]{hyperref}
\usepackage{pxjahyper}

\usepackage{amssymb}
\usepackage{amsmath}
\usepackage{latexsym}
\usepackage{overcite}


%jsbook を report っぽくするスタイルファイル
\usepackage{book2report}
%定理,補題,系,例題,証明などや英語用の定義がされています.
%自分なりにいじってください.
\usepackage{thesis}
% 具体的には以下のように定義されています.
% 英語の定理環境
%  \newtheorem{theorem}{Theorem}[chapter]
%  \newtheorem{lemma}{Lemma}[chapter]
%  \newtheorem{proposition}{Proposition}[chapter]
%  \newtheorem{corollary}{Corollary}[chapter]
%  \newtheorem{definition}{Definition}[chapter]
%  \newtheorem{example}{Example}[chapter]
%  \newtheorem{proof}{Proof}
% 日本語の定理環境
%  \newtheorem{theorem}{定理}[chapter]
%  \newtheorem{lemma}{補題}[chapter]
%  \newtheorem{proposition}{命題}[chapter]
%  \newtheorem{corollary}{系}[chapter]
%  \newtheorem{definition}{定義}[chapter]
%  \newtheorem{example}{例}[chapter]
%  \newtheorem{proof}{証明}
% 証明には番号をつけず,最後は Box で終わります.

% 英語で,見出しのフォントが気に入らなかったら
%\renewcommand{\headfont}{\bfseries}

% ページ数が少ないときはここを大きくしてごまかそう!!効果絶大!!
\renewcommand{\baselinestretch}{1.0}

\begin{document}
\chapter{緒論}
\label{ch:1}
 \section{研究背景と目的}
 \label{sec:1_1}

 \par
 タクシー業界は道路運送法の下で様々な規制がかけられていた.
 しかし,2002年に道路運送法が改正され,規制緩和が行われた.
 そのため,タクシー会社の新規参入が増え,都市部でのタクシーの供給が増えた.
 また,名古屋ではタクシーの自動運転による実証実験が行われている.
 こうした状況では,データに基づく配車や運行の方法を考えることは重要である.

 \par
 一方,近年では通信環境が整備され,プロセッサーの性能が向上し,通信用チップが安価に入手できるようになった.
 つまり,大量のデータを観測,収集し,解析することが容易になった.
 そのため,サイバーフィジカルシステムの考え方に基づく制御が注目を浴びている.

 \par
 本論文ではタクシー乗務員の運行をサポートするシステムと,合理的な運行をするための制御器を提案する.
 また,その制御器の有効性を個々のドライバーが貪欲に運行した場合と比較を行うことによって示す.

 \section{論文の構成}
 \label{sec:1_2}

 \par
 本報告の構成について述べる.
 第2章では,タクシー業界の現状について説明を行い,私達が利用できるデータと提案するシステムについて述べる.
 第3章では,タクシーの移動モデルを混合論理ダイナミカルシステムでモデル化する.
 そして,そのモデルを用いたモデル予測制御法を提案し,その有効性と計算時間にかかる時間を示す.
 第4章では,提案システムで実装したニューラルネットワークを用いた需要予測について述べ,数値評価を行った結果を示す.
 最後に,第5章では結論と今後の課題について述べる.

 \include{end}
