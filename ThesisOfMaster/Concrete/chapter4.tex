%卒業論文用雛形
\documentclass[a4j,12pt,oneside,openany]{jsbook}
% 英語なら以下を使う.
%\documentclass[a4j,12pt,oneside,openany,english]{jsbook}

\usepackage[dvipdfmx]{graphicx}
\usepackage[dvipdfmx]{hyperref}
\usepackage{pxjahyper}

\usepackage{amssymb}
\usepackage{amsmath}
\usepackage{latexsym}
\usepackage{overcite}


%jsbook を report っぽくするスタイルファイル
\usepackage{book2report}
%定理,補題,系,例題,証明などや英語用の定義がされています.
%自分なりにいじってください.
\usepackage{thesis}
% 具体的には以下のように定義されています.
% 英語の定理環境
%  \newtheorem{theorem}{Theorem}[chapter]
%  \newtheorem{lemma}{Lemma}[chapter]
%  \newtheorem{proposition}{Proposition}[chapter]
%  \newtheorem{corollary}{Corollary}[chapter]
%  \newtheorem{definition}{Definition}[chapter]
%  \newtheorem{example}{Example}[chapter]
%  \newtheorem{proof}{Proof}
% 日本語の定理環境
%  \newtheorem{theorem}{定理}[chapter]
%  \newtheorem{lemma}{補題}[chapter]
%  \newtheorem{proposition}{命題}[chapter]
%  \newtheorem{corollary}{系}[chapter]
%  \newtheorem{definition}{定義}[chapter]
%  \newtheorem{example}{例}[chapter]
%  \newtheorem{proof}{証明}
% 証明には番号をつけず,最後は Box で終わります.

% 英語で,見出しのフォントが気に入らなかったら
%\renewcommand{\headfont}{\bfseries}

% ページ数が少ないときはここを大きくしてごまかそう!!効果絶大!!
\renewcommand{\baselinestretch}{1.0}

\begin{document}
\chapter{結論}
\label{ch:5}
\par
本論文では,タクシーの走行支援システムを企業と共同開発し,実装を行った結果を示した.
過去の乗車実績に基づいてマップ上に立てたピンについては,周囲にピンが立っていないセルでも需要が多く発生していることがわかった.
これは,過去の乗車実績をただ表示するだけでなく,需要予測を行いその値に基づいて表示するピンを選ばなければいけないことを示している.
推奨される走行方向の計算では,まず,タクシーの移動モデルを混合論理動的システムを用いて表し,有限区間最適制御問題を作成した.
そして,モデル予測制御を応用して,混合性数計画問題を解くことで空車の最適移動分布を求めた.
推奨される走行方向は最適移動分布の中でもっとも大きい値を持つ方向とした.

 実装にあたっては,大阪のタクシー事業者から実データをいただき,実際の空車分布から実証実験を行った.
そのとき,利己的な進行方向と,推奨される方向とが必ずしも一致しなかった.
これから,推奨される方向を提示して支援を行うことが重要であることがわかる.
混合性数計画問題はNP困難な問題だが,対象領域の設定や分割の仕方を適切に行えば,推奨される方向の提示は実用可能であることが判明した.

\par
また,今回は1単位時間に進むことができるセルは隣接するセルのみとした.
長さを短くし,隣接するセル以外にも空車が移動できるようにすれば,より精度よく最適移動方向を計算できるが,計算時間が急速に増大する.
今後の課題としては,混合整数計画問題の計算の並列化,最適解の近似などによる計算時間の短縮が挙げられる.

\par
現在,車の自動運転に関して多くの研究が行われている.
いずれ,タクシーの自動運転が実用化される可能性があるため,タクシーの最適配車問題は今後も様々な研究がなされていくと考える.
本研究が,これらの研究の発展に寄与することを期待する.

 \include{end}