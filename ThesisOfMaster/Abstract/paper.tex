\documentclass[a4j,10pt,twocolumn]{paper}
\usepackage{graphicx}
\usepackage[varg]{txfonts}
%%% \begin{document}の前に,各エントリーを記述する

\title{タクシーの運転支援システム構築に関する研究}	% 論文のタイトル
\author{広本 将基}		% 著者
\studentid{29C15073}	% 学籍番号
\lab{潮}		% 研究室名

% 英語なら以下2行を定義
%\englishtitle
%\jptitle{日本語の題名}  % 日本語のタイトル

\begin{document}
\absttitle		% 表題の出力

\section{緒論}
流しのタクシーが効率よく乗客をのせるための支援システムの開発は,運転手の待遇改善に繋がる重要な課題である.
本報告では,過去の乗車データと現在の流しのタクシーの分布から最適な進行方向を決定する方法を提案する.
対象領域をいくつかの部分領域に分割し,各部分領域での需要予測をもとに部分領域ごとの流しのタクシーの変化を混合論理ダイナミカルシステムを使ってモデル化する.
モデル予測を応用して,各部分領域でのタクシーの最適移動分布を求めることで最適な進行方向を決定する.
\section{問題の定式化}
対象領域を$N$個の部分領域(あ)(以下セルと呼ぶ)に分割する.
時刻$k$のときのセル$i$($i = 1, 2, \ldots, N$)での空車数を$x_i(k)$とおく.
\section{シミュレーション結果}

%\bibliographystyle{junsrt}
%\bibliography{refs}
\begin{thebibliography}{1}
\bibitem{}
\end{thebibliography}
\newpage
\pagebreak
\end{document}
