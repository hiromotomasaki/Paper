\documentclass[paper]{ieicej}
%\documentclass[invited]{ieicej}% 招待論文
%\documentclass[survey]{ieicej}% サーベイ論文
%\documentclass[comment]{ieicej}% 解説論文
%\usepackage[dvips]{graphicx}
%\usepackage[dvipdfmx]{graphicx,xcolor}
\usepackage[T1]{fontenc}
\usepackage{lmodern}
\usepackage{textcomp}
\usepackage{latexsym}
%\usepackage[fleqn]{amsmath}
%\usepackage{amssymb}

\setcounter{page}{1}

\field{a}
\jtitle{a}
\etitle{a}
\authorlist{%
 \authorentry{広本 将基}{Masaki HIROMOTO}{}\MembershipNumber{}
 %\authorentry{和文著者名}{英文著者名}{所属ラベル}\MembershipNumber{}
 %\authorentry[メールアドレス]{和文著者名}{英文著者名}{所属ラベル}\MembershipNumber{}
 %\authorentry{和文著者名}{英文著者名}{所属ラベル}[現在の所属ラベル]\MembershipNumber{}
}
\affiliate[]{}{}
%\affiliate[所属ラベル]{和文所属}{英文所属}
%\paffiliate[]{}
%\paffiliate[現在の所属ラベル]{和文所属}

\begin{document}
\begin{abstract}
 %和文あらまし 500字以内
 \par

 タクシー業界は道路運送法の下で様々な規制がかけられていた.
 しかし,2002年に道路運送法が改正され,規制緩和が行われた.
 そのため,タクシー会社の新規参入が増え,都市部でのタクシーの供給が増えた.
 また,名古屋ではタクシーの自動運転による実証実験が行われている.
 こうした状況では,データに基づく配車や運行の方法を考えることは重要である.
 \par
 一方,近年では通信環境が整備され,プロセッサーの性能が向上し,通信用チップが安価に入手できるようになった.
 つまり,大量のデータを観測,収集し,解析することが容易になった.
 そのため,サイバーフィジカルシステムの考え方に基づく制御が注目を浴びている.
 \par
 本論文ではタクシー乗務員の運行をサポートするシステムと,合理的な運行をするための制御器を提案する.
\end{abstract}
\begin{keyword}
%和文キーワード 4〜5語
混合整数計画問題,サイバーフィジカルシステム
\end{keyword}
\begin{eabstract}
%英文アブストラクト 100 words
\end{eabstract}
\begin{ekeyword}
%英文キーワード
\end{ekeyword}
\maketitle

\section{まえがき}

\section{定式化}

\ack %% 謝辞

%\bibliographystyle{sieicej}
%\bibliography{myrefs}
\begin{thebibliography}{99}% 文献数が10未満の時 {9}
\bibitem{}
\end{thebibliography}

\appendix
\section{}

\begin{biography}
\profile{}{}{}
%\profile{会員種別}{名前}{紹介文}% 顔写真あり
%\profile*{会員種別}{名前}{紹介文}% 顔写真なし
\end{biography}

\end{document}
